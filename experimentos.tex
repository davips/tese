%%%%%%%%%%%%%%%%%%%%%%%%%%%%%%%%%%%%%%%%%%%%%%%%%%%%%%%%%%%%%%%%%%%%
\subsection{Experimentos}\label{sec:ferramentas}

> 1 - que nao tem free lunch, pois cada estrategia eh a melhor para
> alguns datasets, nao para todos
>
> 2 - explicar pela arvore quais os nichos de cada estrategia. Voce pde
> tambem usar clustering para mostrar que estrategias se adequam a
> nichos parecidos
>
> 3 usar meta-learning (usando a arvore do item 2) para prever a melhor
> estrategia para um novo conjunto de dados
>
> A ordem dos itens 2 e 3 pode mudar. O normal eh o 3 vir antes do 2




\green{baseline p/ AccBal � $�1/|Y|$}

\green{manter apenas estrat�gias que ganharam em algum momento/nicho?}

\red{o tempo ficou sempre abaixo de 1s, ent�o est� dentro do tolerable waiting time.
Posso adotar uma �nica base enorme s� para a compara��o de tempo}

\blue{manter apenas a melhor combina��o learner-dataset, assim � poss�vel o avaliar estrat�gias num cen�rio em que se sabe de antem�o qual � o melhor learner.
manter todas as combina��es serve para o cen�rio em que nada se sabe do dataset, ou seja, � o mais esperado, pois normalmente � preciso ter r�tulos suficientes para se testar classificadores;
al�m disso, quand ose sabe o melhor classificador (a n�o ser por propriedades dos dados: ex.: nominal->NB, muitos atributos->SVM etc.)
� por meio de um problema parecido onde as estrat�gias podem ser testadas}

\blue{plotar (log?) rnd (e clu?) p/ nb,c45 e 5nn at� o fim do pool ou at� atingir passiva, assim serve de panorama do qto as outras estrategias/learners atingiram da ALC possivel)}

\green{rodar eeg-eye-state pra medir tempo nas principais estrategias/learners, pra ser a unica com nintera; rodar como todas as outras: 5x5-fold}

\subsubsection{Afinidade estrat�gia-aprendiz}
\green{�rvore do vencedor accbal e variancia}

\green{�rvore do perdedor accbal e variancia}

\subsubsection{Comparativo}
\green{friedmenyi de  accbal e variancia}

\green{plotar accBal X budgets p/ cada uma das 15 bases com menos de 200 exemplos?}

\green{plotar p/ accBal (apenas nas 85 bases com mais de 200 exemplos?)): \#topos-de-rank X budgets $[10;200]$}

\subsubsection{Aprendizado meta-ativo}
\green{�rvores do vencedor com empate 1-vs-n accbal e variancia}

\red{uma base n�o pode aparecer no cjt de treino e tb no de teste}


