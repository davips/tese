% \afterpage{\clearpage\begin{landscape}
\begin{figure}
\tikzset{
every node/.style={font=\scriptsize,black, thin},
decision/.style={shape=rectangle, minimum height=1cm, text width=1.7cm,
text centered, rounded corners=1ex, draw},
outcome/.style={ shape=rectangle, fill=gray!15, draw, text width=2.8cm, text justified},
decision tree/.style={sibling distance=3cm, level distance=4cm},
cond/.style={blue, yshift=-2mm, shape=rectangle, text centered},
}
\begin{center}
\caption{Distribuição das vitórias dos pares estratégia-algoritmo.
\textit{Cada colocação entre as três melhores é contabilizada como vitória.
% As $25\%$ menos frequentes são representadas pela categoria ``outras''.
}
}
\scalebox{0.9}{
\begin{tikzpicture} [edge from parent/.style={->,above,draw,sloped,midway,gray!30,ultra thick},
text width=2.7cm, align=flush center, grow cyclic,
level 1/.style={level distance=1.8cm,sibling angle=180},
level 2/.style={level distance=3.7cm,sibling angle=90},
level 3/.style={level distance=3.75cm,sibling angle=70},
level 4/.style={level distance=3cm,sibling angle=120},
level 5/.style={level distance=3cm,sibling angle=45},
]

\input treeParesKappa

\label{treeALCKappa}
\end{tikzpicture}
}
\end{center}
\end{figure}
% \end{landscape}\clearpage}