% \chapter{Considerações Finais}
\chapter{Conclusão}\label{conclusao}

\esb{unifica resumidamente os resultados de maneira coerente}

desdobramentos futuros
Foi visto que em 69\% (65\% pra 50qs e 88bs) das vezes
(embora possivelmente incluindo pares inadequados strat-dataset,
mas isso não é um problema sério) (65\% se todos aprendizes forem comparados com o classif do Rnd e 83\%
se for um classif aleatório)
a desvinculação é proveitosa.
No futuro, é possível propor uma estratégia sincretista, que reavalie a adequação do aprendiz
aos dados a cada nova consulta.
% tentei brevemente isso, mas não deu certo; é preciso investigar melhor a quantidade inicial ideal de queries para
% iniciar as avaliações.
Provavelmente, a afinidade apren-strat é mais importante que isso,
por exemplo, Mar vai bem com RFw, mesmo que o classificador ideal para a base seja NB.
Assim, pode ser necessário desvincular não só após a rotulação, mas durante o aprendizado,
caso predições sejam solicitadas. Elas não seriam enviadas ao aprendiz, mas sim ao
classificador apropriado ao momento.
% talvez por isso tenham dado errado as tentativas breves acima.