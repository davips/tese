\section{Máquinas Extremas}\label{elmorig}
% von Zuben:
% ftp://ftp.dca.fee.unicamp.br/pub/docs/vonzuben/ia353_1s13/topico4_P4_1s2013_view.pdf
% Ele coloca uma coluna adicional de 1s em H.
% Considerando que a matriz H tenha posto completo
Uma \ing{\elm}{Extreme Learning Machine} (ELM) é uma rede, não
necessariamente neural e com treinamento diferenciado, que tem recebido crescente
atenção devido à sua simplicidade e performance, comparável com o estado da arte
em classificação \citep{journals/tsmc/HuangZDZ12}.
Sua topologia é a mesma de um \ing{Perceptron multicamadas}{MultiLayer Perceptron} - MLP
\citep{haykin2004comprehensive}, mas com a quantidade de camadas ocultas fixada em apenas uma.
Essa topologia é chamada de \ing{rede de camada oculta única com direcionamento entrada-saída}
{Single Hidden Layer Feedforward Network}
(SLFN).
Com relação à configuração dos neurônios, a ELM é desprovida do peso de viés na
camada de saída e aceita funções de ativação não diferenciáveis na camada oculta.
A camada oculta é gerada com pesos aleatórios e mantida sem alterações
durante o treinamento.
Essa ideia remonta ao trabalho de \cite{rosenblatt1961principles} que optou
por ajustar apenas as últimas camadas.
Na ELM, o ajuste tira proveito do fato de que a camada de saída
é um modelo de regressão linear.
Esse tipo de modelo tem sido estudado desde o século XIX \citep{legendre1805nouvelles},
logo possui uma vasta literatura de métodos que permitem alterar a rede de
forma otimizada conforme explorado a seguir e nas seções
\ref{press}, \ref{emelm} e \ref{oselm}.

A ELM enquanto conceito existe desde o trabalho de \cite{journals/tnn/HuangB98}
onde é proposta a busca direta pelo valor dos pesos de saída sem a necessidade de iterações
e o uso de qualquer função de ativação limitada não linear.
Neste trabalho, dentre os dois tipos de nó mais frequentemente usados,
aditivo logístico sigmoide e função de base radial \citep{journals/tsmc/HuangZDZ12},
optou-se pelo primeiro.
Assim, a função preditiva de uma SLFN com $L$ neurônios ocultos pode ser representada
como segue na Equação \ref{eqelm}.
\begin{equation} \label{eqelm}
f(\bm{x}_{(t)})= \argmax_{\bm{y} \in Y}{\displaystyle\mathop{\sum} _{l=1}^{L}\beta_{l,c(\bm{y})}
g(\bm{a}_l \bm{x}_{(t)} + b_l)} = \bm{y}_{(t)}
\end{equation}
% }, \quad j=1, \cdots, {\it N}.

$\beta_{l,o}$ é o peso da sinapse que conecta o neurônio oculto $l$ ao neurônio de saída $o$;
$c(\bm{z})$ é uma função auxiliar que retorna o índice correspondente à classe
representado pelo vetor $\bm{z} \in Y$;
$g$ é, neste trabalho, a função sigmoide logística;
$\bm{w}_l$ é o vetor de pesos do neurônio $l$ com cada valor $w_i$ representando o peso entre a
entrada $i$ e o neurônio $l$ e, sendo o produto interno deste neurônio uma equação de reta no
espaço de parâmetros, $b_l$ é o valor de seu viés de deslocamento em relação à origem.
Mais detalhes da formulação da primeira camada podem ser consultados na literatura sobre o
Perceptron multicamadas convencional \citep{haykin2004comprehensive}.
Se $\beta$ for adotada como uma matriz, a equação \ref{eqelm} pode ser escrita compactamente da
forma (Equação \ref{eq2}):
\begin{equation} \label{eq2}
H \beta = T
\end{equation}

onde $H$ é a \ing{matriz de saídas da camada oculta}{hidden layer output matrix} da SLFN,
ou seja, cada coluna corresponde à saída de um neurônio oculto e cada linha corresponde a um
exemplo do conjunto de treinamento;
$\beta$ é a matriz contendo os pesos que conectam a camada oculta à camada de saída, ou seja, cada
coluna corresponde a um neurônio de saída e cada linha corresponde a um neurônio oculto;
e $T$ é a matriz objetivo, que contém em cada linha a
representação binária (um-de-n) $\bm{y}$ da classe de cada exemplo.
O cálculo de $\beta$, que é análogo a treinar a rede,
pode ser feito por meio do cálculo da pseudoinversa $H^{\dagger}$ de $H$
\citep{rao1971generalized}:
\begin{equation} \label{eq3}
\beta = H^{\dagger} T
\end{equation}
Essa abordagem tem a vantagem de ser também uma minimização da norma dos pesos -
propriedade que reduz a possibilidade de sobreajustamento aos dados de treinamento
\citep{journals/tit/Bartlett98}.
Uma vez calculada a pseudoinversa,
novos exemplos podem alimentar a entrada da rede e gerar predições conforme a função $f$
presente na Equação \ref{eqelm}.

% https://reference.wolfram.com/mathematica/tutorial/LinearAlgebraMatrixComputations.html
% explica bem a pinv

% \input topologia
% 
% \subsection{PRESS}\label{press}
% % Meta-learning to optimize the number of hidden nodes of MLP networks trained
% % by Extreme Learning Machine algorithm (2011)
% % http://www.mendeley.com/download/public/1683461/5291097694/743b3ac1a6f01b8b9ad380254dc9ee7de2146086/dl.pdf
% % \citep{prud2011}
% 
% Um desses métodos, chamado
% \textit{soma dos quadrados dos erros de predição} - PRESS\footnote{\textit{PREdiction Sum of Squares}}
% \citep{myers2000classical,allan1974relationship}, permite estimar o erro de generalização do modelo.
% Exceto nos casos de instabilidade numérica, essa estimativa é exata e equivalente à que seria obtida
% pela validação cruzada via \textit{Leave-One-Out} (LOO), representando uma redução na complexidade
% computacional em relação à quantidade de treinamentos de linear para constante - ou
% de quadrática para linear, se for considerada a quantidade de exemplos.
% 
% % http://www.jstor.org/stable/2686028?__redirected
% % o começo fala do PRESS, com formula clarificando(?)
% 
% % (see [15] and [16] for details of this formula and its implementations)
% % 15:Classical and Modern Regression With Applications,
% % 16:?Recursive lazy learning for modeling and control,? http://www.researchgate.net/publication/2301623_Recursive_Lazy_Learning_for_Modeling_and_Control/file/79e4151136c53c3067.pdf
% 
% .
% \ano{fórmula hat}
% 
% 
% % \ano{PRESS: erro estatistico (de regressão?)}
% % 
% % \ano{quadrado do resíduo: erro I e CI?}
% % 
% % diferencia 'statistical error' e 'residual':
% % 
% % {http://en.wikipedia.org/wiki/Errors\_and\_residuals\_in\_statistics}
% % 
% % error (or disturbance) of an observed value is the deviation of the observed value from
% % the (unobservable) true function value
% % residual of an observed value is the difference between the observed value and the
% % estimated function value.
% 
% \subsection{EM-ELM} \label{emelm}
% 
% crescimento incremental da
% \ing{\elm por minimização do erro}{Error Minimized Extreme Learning Machine} -
% EM-ELM \citep{journals/tnn/FengHLG09}.
% Além de permitir o crescimento da rede sem recalcular a pseudoinversa,
% a EM-ELM permite o cálculo da PRESS a um custo reduzido.
% 
% % * EM-ELM serve de base para várias variantes de ELM: AIE-ELM, D-ELM, CEOS-ELM
% % survey simples: http://www.wisdombasedcomputing.com/vol1issue1april2011/paper4.pdf
% 
% \tar{colocar fórmula que obtém PRESS da EM-ELM}
% \ano{ver se k é usado pra alguma outra coisa}
% Sendo $\bm{h}_{(k)}$ o vetor coluna de valores de função de ativação correspondente
% à coluna $k$ da matriz $H_k$ na iteração $k$, ou seja, após o acréscimo de $k$
% neurônios,
% a matriz $H_k$ pode ser representada conforme a Equação \ref{eqh}.
% \begin{equation}\label{eqh}
% \mathbf{H}_k = [\mathbf{H}_{k-1}, \bm{h}_k]
% \end{equation}
% Assim, cada novo neurônio faz a matriz $H$ aumentar em uma coluna.
% A atualização dos pesos $\beta$ se dá por meio das equações
% \ref{em1}, \ref{em2} e \ref{em3}.
% \begin{equation}\label{em1}
% \mathbf{D}_k=\frac{\bm{h}_k(\mathbf{I}-\mathbf{H}_k\mathbf{H}_k^\dagger)}
% {\bm{h}_k(\mathbf{I}-\mathbf{H}_k\mathbf{H}_k^\dagger)\bm{h}_k}
% \end{equation}
% \begin{equation}\label{em2}
% \mathbf{U}_k=\mathbf{H}_k^\dagger(\mathbf{I}-\bm{h}_k\mathbf{D}_k)
% \end{equation}
% \ano{procurar papel ou código que mostre as fórmulas siomplificadas para
% ficar sem redundância}
% \begin{equation}\label{em3}
% \beta_{k+1}=\mathbf{H}_{k+1}^\dagger\mathbf{T}=\left [
% \begin{matrix}
% {\bf U}_{k}\cr {\bf D}_{k} 
% \end{matrix}
% \right] \mathbf{T}
% \end{equation}
% 
% 
% 
% 
% 
% 
% \subsection{OS-ELM}\label{oselm}
% \ano{OS-ELM vale a pena porque evita uma inversa grande, calcula apenas uma vez e favorece o uso do pequeno cache.}
% 
% Em sistemas interativos, frequentemente é adotado o esquema incremental de aprendizado.
% O tempo de treinamento é reduzido quando é possível aproveitar o resultado de cálculos
% prévios durante todo o processo.
% % ROS-ELM A Robust Online Sequential Extreme Learning Machine 2007:
% % http://download.springer.com/static/pdf/966/chp%253A10.1007%252F978-3-540-72383-7_126.pdf?auth66=1398531288_8f1d69c65bc2dcd81b7a7451031a0dad&ext=.pdf
% % Na OS-ELM, os biases precisam ser escolhidos dentro de determinados intervalos, dependentes de dataset, para evitar ill-conditioning/singularity of H.
% % Eles calculam, aparentemente, os biases pela diagonal de XW (X: att values? W: pesos?).
% No caso da ELM, isso é possível por meio da
% \ing{\elm sequencial}{On-line Sequential Extreme Learning Machine} (OS-ELM)
% proposta por \cite{conf/iastedCI/HuangLRSS05}.
% A técnica é baseada no algoritmo dos mínimos quadrados recursivo
% que utiliza a fórmula de Sherman-Morrison-Woodbury
% % http://books.google.com.br/books?hl=en&lr=&id=iD5s0iKXHP8C&oi=fnd&pg=PT15&dq=+An+introduction+to+optimization&ots=3PqthZAq8d&sig=Befcr8te239chWT6UOVyVrMIkSo#v=snippet&q=recursive%20least&f=false
% \citep{chong2013introduction}.
% $H$ passa a ser considerada em função do tempo - indicado subscrito
% para melhor legibilidade: $H_{(t)}$.
% Para ser possível encontrar a solução dos mínimos quadrados de
% $H_{(0)} \beta_{(0)} = T_{(0)}$,
% que se refere ao conjunto inicial de dados rotulados (instante $t=0$),
% a pseudoinversa esquerda $H^\dagger_{(0)}$ deve ser calculada por:
% 
% \begin{equation} \label{eq4}
% P_{(0)}=(H^\top_{(0)}H_{(0)})^{-1}
% \end{equation}
% \begin{equation} \label{eq5}
% H^{\dagger}_{(0)} = P_{(0)}H^\top_{(0)}
% \end{equation}
% 
% Assim, é possível realizar um treinamento inicial da rede.
% Para os lotes de dados subsequentes, ou seja,
% quando $t>0$, apenas $\beta_{(t)}$ e $P_{(t)}$ precisam ser mantidos, onde:
% 
% \begin{equation} \label{eq6}
% P_{(t)} = P_{(t-1)} - P_{(t-1)} H^{\top}_{(t)} (I + H_{(t)} P_{(t-1)} H^{\top}_{(t)})^{-1} H_{(t)}
% P_{(t-1)}
% \end{equation}
% \begin{equation} \label{eq7}
% \beta_{(t)} = \beta_{(t-1)} + P_{(t)} H^{\top}_{(t)} (T_{(t)} - H_{(t)} \beta_{(t-1)})
% \end{equation}
% 
% O cálculo incremental de $\beta_{(t)}$ e $P_{(t)}$ evita os custos de se recalcular a
% pseudoinversa para cada novo lote de dados.
% 


\subsection{I-ELM}\label{ielm}
Apesar da vantagem do cálculo da pseudoinversa ser uma etapa única no
aprendizado da ELM, trata-se de um cálculo que pode ser custoso.
Uma alternativa é a \elm \textit{incremental} (I-ELM) proposta por
\cite{journals/tnn/HuangCS06}.
O aprendizado da I-ELM começa com um neurônio e
progride com a adição de novos neurônios, um a um.
O valor do peso é calculado de forma a reduzir o erro entre o
valor predito e o valor esperado.
Cada neurônio adicionado na iteração $t$ requer um novo peso $\beta_{1,t}$
para conectá-lo ao neurônio de saída.
Assumindo, por simplicidade, apenas um atributo preditivo,
a matriz alvo $T$ se torna um \textit{vetor coluna}, representado aqui por $\bm{\tau}$.
\ano{ver se ja mencionei h antes, na parte da EM-ELM}
Sendo $\bm{h}_{(t)}$ o vetor de valores da função de ativação do novo neurônio,
ou seja, com os valores da coluna $t$ de $H$,
o valor do peso é calculado segundo a Equação \ref{eq:ielm}.
\begin{equation}\label{eq:ielm0}
\bm{e}_{(0)} = \bm{\tau}
\end{equation}
\begin{equation}\label{eq:ielm}
\beta_{1,t}=\frac{\bm{e}_{(t-1)}\cdot \bm{h}_{(t)}}
{\bm{h}_{(t)}\cdot \bm{h}_{(t)}}
\end{equation}
Onde $\bm{e}_{(t)}$ é o vetor de erros residuais antes da inserção do
novo neurônio na iteração $t$.
Todos os vetores têm cada uma de suas componentes associadas a um dos exemplos do conjunto de treinamento.
A predição da classe de novos exemplos se dá da mesma forma que na ELM original.

\subsection{CI-ELM}
Uma proposta similar à I-ELM, chamada \textit{convexa incremental} (CI-ELM),
opta pelo reajuste de todos os pesos da camada de saída a cada novo neurônio
acrescentado \citep{journals/ijon/HuangC07}.
Diferentemente da I-ELM, ela é baseada na otimização convexa de Barron
\citep{journals/tit/Barron93}, mas mantém a capacidade de redução do erro
treinamento de forma monotônica até qualquer valor arbitrariamente pequeno.
Ela é capaz de atingir uma convergência maior que a I-ELM
para um mesmo número de iterações.
Assim, dada uma meta de erro, a CI-ELM possibilita a indução de redes mais compactas.

Para cada novo neurônio, seu peso é calculado de forma análoga à apresentada na
Seção \ref{ielm}, porém os valores de saída do novo neurônio representados por
$\bm{h}_{(t)}$ são substituídos pela diferença $\bm{d}_{(t)}$
entre o erro corrente e o erro devido ao novo neurônio,
conforme equações \ref{eq:cielm} e \ref{eq:cielm2}.
\begin{equation}\label{eq:cielm}
\bm{d}_{(t)}=\bm{e}_{(t-1)} - (\bm{\tau} - \bm{h}_{(t)})
\end{equation}
\begin{equation}\label{eq:cielm2}
\beta_{1,t}=\frac{\bm{e}_{(t-1)}\cdot \bm{d}_{(t)}}
{\bm{d}_{(t)}\cdot \bm{d}_{(t)}}
\end{equation}
% Onde $T$ é o vetor alvo, ou seja, contém os valores do atributo preditivo.
A etapa adicional de reajuste é feita para todo o conjunto de pesos anteriormente
acrescentados ($\{\beta_i \mid 1 \leq i < t\}$)
de acordo com a Equação \ref{reajuste}.
\begin{equation}\label{reajuste}
\beta_{1,i}^* = (1 - \beta_{1,t}) \cdot \beta_{1,i}
\end{equation}
Assim como na I-ELM, a CI-ELM realiza as predições por meio da fórmula da Equação \ref{eqelm}.

% 
% \ano{citar as outras ELMs que crescem ou aprendem incrementalmente?}
% % Além da OS-ELM, outras variantes da ELM têm sido propostas:
% % ROS-ELM, que evita matrizes mal-condicionadas por meio do ajuste dos vieses $b_i$ \cite
% % {conf/isnn/HoangHVW07};
% % EI-ELM ou EM-ELM, não-incrementais, que fazem a rede crescer \textit{congelando}
% % \citep{huang2008enhanced} ou \textit{atualizando} \citep{journals/tnn/FengHLG09}
% % nós antigos;
% % e, CEOS-ELM, que faz a rede crescer durante o aprendizado \textit{on-line} \citep{conf/ijcnn/LanSH09}.
% %
% % Para o presente trabalho, a OS-ELM ``canônica'' foi adotada, pois cada variante tem seus próprias
% % vicissitudes e o propósito deste trabalho é apenas dar uma primeira impressão a respeito da
% % performance da ELM com diferentes estratégias.
% % Também é importante lembrar que, na presença de uma matriz $H_{(0)}$ mal-condicionada,
% % OS-ELM não converge para ELM.
% 
% \tar{Citar von Zuben e outros brasileiros}
% 
% 
% \subsection{Configuração adotada para aprendizado ativo}
% \ano{mover isso pra algum lugar melhor na tese}
% Até onde o conhecimento do autor permite alcançar,
% não existe pesquisa em aprendizado ativo que enfoque ELMs.
% As poucas abordagens encontradas usam aprendizado ativo apenas como uma ferramenta
% sem maiores desenvolvimentos em torno do assunto em si.
% 
% \subsubsection{Aprendizado ativo e \elms}\label{aelmrev}
% Um exemplo é a \ing{consulta por amostragens e desvio padrão}{standard deviation query-by-bagging}
% \cite{conf/his/AyerdiMG12},
% que é conceitualmente próximo à \textit{amostragem por incerteza}
% e \textit{consulta por comitê}.
% É uma abordagem genérica no sentido de que ela pode ser aplicada em
% um comitê de modelos gerados por qualquer algoritmo de aprendizado.
% Outro exemplo é o \ing{aprendizado ativo baseado em fixação}{fixation-based active learning},
% que treina uma ELM para ser usada como uma \ing{tabela de mapeamento de cores}{color look-up table}
% para segmentação.
% Na parte de aprendizado ativo, é calculada a entropia de um histograma feito das intensidades
% em uma região circular \cite{journals/soco/PanPLW12} - novamente,
% o aprendizado ativo é usado apenas como uma ferramenta independente do classificador adotado.
% Outro, menos relacionado exemplo,
% é a incerteza de um \ing{arcabouço}{framework} bayesiano não-paramétrico baseado em ELMs
% \cite{conf/ictai/ChatzisKD11}.
% À parte desses trabalhos isolados,
% nenhum estudo foi encontrado na literatura de aprendizado de máquina.
% Consequentemente, estudos comparativos abrangentes estão ausentes da literatura.
% 
% \subsubsection{Adequação da \elm para aprendizado ativo}\label{adeq}
% O aprendizado ativo é um caso de aprendizado interativo.
% Assim, o aprendiz representa um modelo que evolui com o tempo que possa fazer predições
% a qualquer momento.
% Sendo a \elm original uma rede de topologia fixa e treinamento em lote,
% o modelo resultante pode demorar para ser gerado, dependendo da quantidade de exemplos,
% de atributos e de classes. Além disso, algum tipo de seleção de modelos é necessária para
% a busca do valor ideal para $L$.
% O processo de seleção de modelos habitualmente envolve o uso de validação cruzada
% \ano{citar artigo sobre seleção de moldelo, ``pra fazer melhor proveito dos dados''}
% que é custoso, especialmente se envolver a validação menos enviesada que é o LOO.
% \esb{citação}
% 
% Dependendo da aplicação, o oráculo é humano, sujeito ao \textit{tempo máximo tolerável de espera}.
% \ano{vou  definir aqui, ou antes?}
% Dentre as variantes da \elm citadas na seções \ref{topologia}, \ref{oselm} e \ref{emelm}
% a mais conveniente é a combinação OS-ELM, EM-ELM e PRESS,
% por envolver a formulação canônica, ou seja, analiticamente desenvolvida
% por teoremas já consagrados na literatura de álgebra linear.
% % Outra vantagem de se evitar as variantes da Seção \ref{topologia} é a ausência de parâmetros? ops!
% % mais citadas e mais representativas sem pirotecnias
% 
% Uma abordagem similar foi empregada por \cite{journals/ijon/HeeswijkMOL11},
% porém sem o crescimento acelerado que poderia ser proporcionado pela EM-ELM
% e sem a possibilidade de aprendizado incremental.
% Eles recalculam a pseudoinversa a cada novo neurônio acrescentado até encontrar a topologia com
% a menor PRESS.
% % Eles implementaram uma ELM específica para \ing{unidades de processamento gráfico}
% % {Graphics Processing Unit} (GPU).
% % No presente trabalho, exceto para SGmulti, os conjuntos de treinamento não passam de $200$ exemplos,
% % não justificando o uso de GPU, especialmente em vista do custo de transferência de dados.
% 
% 
% \ano{expor o algoritmo}
% 
% \ano{crescer de 1 em 1, mas para acelerar conforme a necessidade incrementar de 1 em 1 exemplo e crescer
% de q em q}
% \esb{testa a faixa de  de L-10 até L+10 neurônios;}
% começando com L=1
