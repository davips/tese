\chapter{Aprendizado Meta-ativo}\label{aml}
\esb{O mais próximo de meta que encontrei na literatura foi a combinação MAB de estratégias:
http://arxiv.org/pdf/1309.6830.pdf}

\tar{parece claro pelos exps anteriores que algumas estratégias simplesmente não aceitam
certos aprendizes (ainda falta explicitar isso lá, apontando caso a caso e justificando
teoricamente, para evitar que seja dito que ocorreu overfitting quando essa info
for usada no metaap),
então isso facilitaria um metaap que indique pares strat/aprendiz,pois
reduziria a quantidade de combinações possíveis}

\ano{ainda é preciso ver se ranking supera ranking medio (default)}

\ano{possibilidades: um experimento para cada learner separadamente;
sortear budget no exp acima;
sortear learners num experimento só;}





\tar{LOO é usado; então para ter confiança est. pode-se adotar o teste de McNemar}

\tar{rodei meta multirrotulo pra vários metaclassif:
c45=0.50 Maj=0.46}

\tar{vou rodar meta monorrotulo pra vários metaclassif com trios lea-bud-dataset sem
roubar:
c45=0.19 Maj=0.14}

\ano{resume livro de meta-ap.; cita Marcilio, statlog;
falar do desafio de nao usar informacoes sobre as classes nas metafeatures
que existe em metaAL e não em metaPassiveL}

\ano{como a predição da melhor não vence a majoritaria preciso usar ranking
(ainda posso testar algumas variações, como fixar
cada um dos learners possíveis)}


\ano{Acho que NFL se refere à impossibilidade de aprendizado sem viés
no caso de ap. ativo trata-se mais de amostragem do que de aprendizado;
as estrategias agnósticas são um exemplo de pura amostragem;
para o AL ser enxergado como um problema de aprendizado é
preciso enxergá-lo como um problema de classificar cada exemplo como
``consultar'' ou ``não consultar'';}

\ano{isso acima poderia até se tornar um ensemble de AL que trabalha em cima de exemplos
transformados pelas diferentes estratégias
- um classificador único que aprenderia de vários datasets
quais exemplos são bons de consultar,
mas infelizmente é totalmente inviável implementar isso agora;
ele seria treinado com a sequencia otima de exemplos-transformados de
cada base: os primeiros seriam os positivos e últimos seriam os negativos.
)}

\tar{meta-aprendizado pra recomendar quando não usar pode ter maior acurácia.}

\ano{diferenciar do meta-aprendizado ativo do Prudêncio}

Multiarmed bandit problem
http://www.ualberta.ca/~szepesva/Thesis/varun08.thesis.pdf

\section{Recomendação de estratégias}\label{recom}
Um sistema de recomendação baseado em meta-aprendizado tem a função de sugerir
o algoritmo e parâmetros mais apropriados, ou um ranqueamento das opções disponíveis,
para uma determinada base de dados \citep{books/daglib/0022052}.
As medidas escolhidas para caracterização das bases foram obtidas da literatura
de meta-aprendizado tanto supervisionado quanto não supervisionado.
Devido à impossibilidade de se prever o intervalo dos valores e definir
o particionamento dele, optou-se por substituir o esquema de histogramas
pelos valores mínimo, máximo, médio e razão entre mínimo e máximo.
Medidas de agrupamento também foram incluídas visando identificar possíveis afinidades
de algumas bases com a estratégia de amostragem por agrupamento.
\begin{table}
\caption{Descrição dos meta-atributos.}
\begin{tabular}{c|c|c}
\textbf{meta-atributo} & \textbf{descrição}&\textbf{fórmula} \\ \hline
\makecell{$\av_{min}$,$\av_{max}$\\$\av_{min/max}$,$\av_{avg}$} &
\makecell{média (mínima, máxima,\\razão entre ambas e média)} & formula? \\ \hline
\makecell{$\sd_{min}$,$\sd_{max}$\\$\sd_{min/max}$,$\sd_{avg}$} &
\makecell{desvio padrão (\textit{idem})} & formula? \\ \hline
\makecell{$\en_{min}$,$\en_{max}$\\$\en_{min/max}$,$\en_{avg}$} &
\makecell{entropia (\textit{idem})} & formula? \\ \hline
\makecell{$\co_{min}$,$\co_{max}$\\$\co_{min/max}$,$\co_{avg}$} &
\makecell{correlação entre pares\\de atributos (\textit{idem})} & formula? \\ \hline
\makecell{$\sk_{min}$,$\sk_{max}$\\$\sk_{min/max}$,$\sk_{avg}$} &
\makecell{assimetria (\textit{idem})} & formula? \\ \hline
\makecell{$\ku_{min}$,$\ku_{max}$,\\$\ku_{min/max}$,$\ku_{avg}$} &
curtose (\textit{idem}) & formula? \\ \hline
\makecell{$\val_{min}$,$\val_{max}$,\\$\val_{min/max}$,$\val_{avg}$} &
\makecell{quantidade de valores\\nominais (\textit{idem})} &
$\val=\{|a| \mid \forall a\in A,\nom(a)=1\}$\\ \hline
% Rnd & \makecell{exploratória\\aleatória} & \makecell{agnóstica}&autônoma&nenhuma&$\mathcal{O}(0)$\\ \hline
\end{tabular}
\end{table}

% "AH-conect.-Y", "AH-Dunn-Y", "AH-silhueta-Y", "AH-conect.-1.5Y", "AH-Dunn-1.5Y",
% "AH-silhueta-1.5Y",
%       "AH-conect.-2Y", "AH-Dunn-2Y", "AH-silhueta-2Y", "kM-conect.-Y", "kM-Dunn-Y",
% "kM-silhueta-Y", "kM-conect.-1.5Y", "kM-Dunn-1.5Y",
%       "kM-silhueta-1.5Y", "kM-conect.-2Y", "kM-Dunn-2Y", "kM-silhueta-2Y").
%    val nonHumanNumAttsNames = "\"#classes\",\"#atributos\",\"#exemplos\"," +
%       "\"#exemplos/#atributos\",\"%nominais\",\"log(#exs)\",\"log(#exs/#atrs)\"," +

metafeat = correl spearm entre queries de: Ageucl, Agmanh e Agmaha

c45    5nn   maj
3buds
0.252 0.227 0.283 qtd de grupos = 94 qtd de metaexemplos: 1974 Winner nenhum*
0.30   0.33  0.283 qtd de grupos = 94 qtd de metaexemplos: 1974 Winner lea
0.276 0.316 0.283 qtd de grupos = 94 qtd de metaexemplos: 1974 Winner 3atts*
0.300 0.347 0.283 qtd de grupos = 94 qtd de metaexemplos: 1974 Winner ambos
2buds
0.246 0.228 0.285 qtd de grupos = 94 qtd de metaexemplos: 1316 Winner nenhum*
0.314 0.327 0.285 qtd de grupos = 94 qtd de metaexemplos: 1316 Winner lea
0.286 0.305 0.285 qtd de grupos = 94 qtd de metaexemplos: 1316 Winner 3atts*
0.307 0.339 0.285 qtd de grupos = 94 qtd de metaexemplos: 1316 Winner ambos
1bud
0.252 0.223 0.305 qtd de grupos = 94 qtd de metaexemplos: 658 Winner nenhum*
0.280 0.321 0.305 qtd de grupos = 94 qtd de metaexemplos: 658 Winner lea
0.324 0.307 0.305 qtd de grupos = 94 qtd de metaexemplos: 658 Winner 3atts*
0.328 0.328 0.305 qtd de grupos = 94 qtd de metaexemplos: 658 Winner ambos
*-> errado, pois gera exemplos repetidos com classes diferentes


\esb{devido à escassez de metaexemplos, LOO foi adotado em conjunto com treinamento
em grupos de exemplos}

\ano{recomendação de estratégia para base,
budget e aprendiz dados [e qtd inicial de rótulos?]}

\ano{escolhe uma dentre várias}

\ano{escolhe cada uma contra random}

\ano{mostra quando random é a melhor (interessante)}

\ano{mostra quando random é a melhor ou empata}

\ano{recomendação de aprendiz
(NB,C45,etc. e ''não-use essa estrat.'' ou ''use aleatório'')
para uma dada estratégia}

\tar{metaap pra decidir entre AG, LU ou TU}

\ano{sucesso garantido: recomendar strat por tempo de consulta estimado.
seria muito óbvio?}

Critério de empate:
Um algoritmo é considerado adequado se ele não for
pior que o melhor de todos com significância estatística
\cite{books/daglib/0022052}.


\tar{mais metafeatures:
 só usa atributos nominais (NB,...)
2- só usa atributos numéricos (SVM, ELM, ...)
3- sofre com atributos irrelevantes (5NN ou medir impacto de adicionar atts irrelevantes)
4- seleciona atributos internamente (árvores)
5- fornece boas estimativas de probabilidade (medir entropia média pra saber)
6- é baseado em distância (5NN)
7- fronteira de decisao horiz e vertical (arvore)}

\ano{\citep{conf/ijcnn/SoutoPSACLS08}, sobre clustering,
usam ranking médio como default:
Method SRC
Default 0.59 +- 0.37
Meta-Leaner 0.75 +- 0.21
;
``melhoram`` statlog, aplicando log}

\citep{kalousis2002algorithm} pg. 43 tem metafeatures (inclusive histograma10,
mas como as faixas ideais são imprevisíveis, resolvi isso colocando max, min, avg e min/max)
Posso colocar ou omitir a informação de distribuição das classes; ela normalmente não
é conhecida na prática.

\section{Meta-estratégia}

\ano{vai que o tempo de phd se multiplica:
testar com outros meta-classificadores; Meta-estratégia dinâmica}

\section{Experimentos e Resultados}
\ano{citar programa R para a extração de parte dos meta-atributos \citep{team2010r}}

\section{Considerações}