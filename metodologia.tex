\chapter{Experimentos e Resultados}\label{cap:experimentos}
Foi empreendida uma avalia��o experimental para comprova��o emp�rica das contribui��es
citadas no Cap�tulo \ref{cap:contrib}.


%%%%%%%%%%%%%%%%%%%%%%%%%%%%%%%%%%%%%%%%%%%%%%%%%%%%%%%%%%%%%%%%%%%%
\section{Metodologia}\label{sec:metodologia}
Para haver solidez estat�stica e tamb�m para uma boa representatividade dos variados dom�nios
existentes em aplica��es reais, foram selecionadas as cem bases de dados mais diversas
do projeto \red{ucipp <- refenciar como?} - por�m apenas noventa e seis bases foram vi�veis com rela��o a tempo de processamento
e puderam ter seus experimentos terminados.
O crit�rio de diversidade envolveu manter apenas uma dentre cada grupo de bases de um mesmo dom�nio.
Com exce��o de algumas que tivessem caracter�sticas (atributos, classes e n�mero de exemplos)
suficientemente diferentes ou cuja acur�cia passiva fosse suficientemente distinta para pelo
menos tr�s classificadores.

As \elms e as estrat�gias baseadas na m�trica de dist�ncia de Mahalanobis precisaram ser
transformadas com a substitui��o de atributos nominais por num�ricos e pela aplica��o do
\ing{filtro de padroniza��o}{z-score}
para possibilitar os c�lculos e para evitar matrizes mal condicionadas - respectivamente.
Os demais algoritmos de aprendizado fizeram eventual uso de seus pr�prios filtros internos
providos pela ferramenta Weka \citep{journals/sigkdd/HallFHPRW09}.

\subsection{Avalia��o}\label{sec:cenario}

\subsubsection{M�tricas}
kappa:\citep{conf/pkdd/Shah11}
accbal: \citep{journals/bmcbi/MassoV10}



%%%%%%%%%%%%%%%%%%%%%%%%%%%%%%%%%%%%%%%%%%%%%%%%%%%%%%%%%%%%%%%%%%%%
\subsection{Ferramentas}\label{sec:ferramentas}
\green{descrever hardware, mysql}

\green{descrever par�metros dos learners}

\green{descrever par�metros das estrat�gias}

xxxxxxxxxxxxxxxxxxxxxxxxxxxxxxxxxxxxxxxxxxx
% algoritmos prontos
Adicionalmente, diversos classificadores base s�o oferecidos como
extens�o da classe \textit{Classifier}.
Um conjunto mais abrangente de classificadores est� dispon�vel na
ferramenta Weka .

\ano{exemplos repetidos foram removidos, mas mantido um deles com a moda dos r�tulos}

Para complementar os testes estat�sticos das ferramentas citadas, o
programa R \citep{team2010r} foi adotado.

%%%%%%%%%%%%%%%%%%%%%%%%%%%%%%%%%%%%%%%%%%%%%%%%%%%%%%%%%%%%%%%%%%%%%
\input experimentos %small; big
