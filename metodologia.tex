\chapter{Experimentos e Resultados}\label{cap:experimentos}

.
mahala, DW que usam mahala e ELMs necessitaram aplica��o de filtro nominal para bin�rio e z-score nos atributos
para evitar matrizes mal condicionadas

%%%%%%%%%%%%%%%%%%%%%%%%%%%%%%%%%%%%%%%%%%%%%%%%%%%%%%%%%%%%%%%%%%%%
\section{Metodologia}\label{sec:metodologia}
\subsection{Cen�rio}\label{sec:cenario}

O cen�rio detalhado adotado neste trabalho � baseado em \pool;
de consulta por classe monorr�tulo; com amostragem um a um de uma distribui��o estacion�ria com
custos uniformes; com or�culo �nico e consistente,
sendo um r�tulo inicial por classe simuladamente obtido via
\engl{aprendizado guiado}{guided learning} \citep{conf/kdd/AttenbergP10};
com atributos sem valores faltantes;
e, sem redu��o do tamanho das bases de dados originais\footnote{Exceto para a estrat�gia de \eer,
para ser computacionalmente trat�vel.}.

%%%%%%%%%%%%%%%%%%%%%%%%%%%%%%%%%%%%%%%%%%%%%%%%%%%%%%%%%%%%%%%%%%%%
\subsection{Ferramentas}\label{sec:ferramentas}
\green{descrever hardware, mysql}

xxxxxxxxxxxxxxxxxxxxxxxxxxxxxxxxxxxxxxxxxxx
% algoritmos prontos
Adicionalmente, diversos classificadores base s�o oferecidos como
extens�o da classe \textit{Classifier}.
Um conjunto mais abrangente de classificadores est� dispon�vel na
ferramenta Weka \citep{hall2009weka}.

Para complementar os testes estat�sticos das ferramentas citadas, o
programa R \citep{team2010r} foi adotado.

%%%%%%%%%%%%%%%%%%%%%%%%%%%%%%%%%%%%%%%%%%%%%%%%%%%%%%%%%%%%%%%%%%%%%
\input experimentos %small; big
