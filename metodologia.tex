\chapter{Metodologia}\label{metodologia}
Foi empreendida uma avaliação experimental para identificar os pontos
críticos na escolha de estratégias e também para comprovação empírica das propostas
dos capítulos \ref{propostas} e \ref{aml}.
Para haver solidez estatística e também para uma boa representatividade dos
variados domínios existentes em aplicações reais, foram selecionadas as
cem bases de dados mais diversas
do projeto de \cite{doi/ucipp} que é baseado no repositório UCI \citep{bache2013uci}.
Porém, apenas noventa e quatro \ano{conferir} bases foram viáveis com relação a
tempo de processamento e puderam ter seus experimentos terminados.
O critério de diversidade envolveu manter apenas uma dentre cada grupo de bases de um
mesmo domínio - com exceção de algumas que tivessem características
(atributos, classes e número de exemplos)
suficientemente diferentes ou cuja acurácia passiva fosse suficientemente distinta para pelo
menos três dos algoritmos de aprendizadomencionados na Seção \ref{learners}.

\ano{ilustrar o uso de banco de dados, a hierarquia de classes das estratégias;
fluxograma de 4 fases: queries, hits, measures, sumarização (médias/friedmans = latex)}

\tar{demsar usa (0.000, +-0.005)  no exemplo}

% Nesse artigo é discutido o fato do Friedman ignorar as diferenças entre modelos (http://www.r-bloggers.com/beware-the-friedman-test/):
% @article{zimmerman1993relative,
%   title={Relative power of the Wilcoxon test, the Friedman test, and repeated-measures ANOVA on ranks},
%   author={Zimmerman, Donald W and Zumbo, Bruno D},
%   journal={The Journal of Experimental Education},
%   volume={62},
%   number={1},
%   pages={75--86},
%   year={1993},
%   publisher={Taylor \& Francis}
% }

\section{Descrição do cenário}
O cenário adotado neste trabalho é baseado em \pool.
Especificamente, há as seguintes restrições de escopo:
\begin{itemize}
 \item consulta pela classe (não por valores de atributos, por exemplo);
 \item monorrótulo;
 \item distribuição estacionária;
 \item custo por erro de classificação uniforme;
 \item custo por consulta uniforme;
 \item oráculo único e constante\footnote{Sempre a mesma resposta para um mesmo exemplo.}
 sujeito a ruído\footnote{Apesar da contradição em atribuir erro a uma entidade
 denominada \textit{oráculo}, essa concessão deve ser feita tanto pelo fato do supervisor
 humano ser sujeito à fadiga quanto pela adoção, nesta tese, de bases de dados reais.};
 \item os domínios dos atributos nominais são previamente conhecidos;
 \item atributos sem valores faltantes;
\end{itemize}
e as seguintes condições que visam rigor experimental, replicabilidade e
verossimilhança com aplicações reais:
\begin{itemize}
 \item consulta \ing{um a um}{on-line};
 \item toda\footnote{Exceto exemplos duplicados.} a base de dados original é utilizada no
   processo de validação cruzada\footnote{Algumas estratégias fazem uma amostragem interna,
   que se renova a cada consulta.};
 \item um rótulo inicial por classe; e,
 \item critério de parada determinado pelo orçamento.
 \end{itemize}
A decisão de um rótulo inicial por classe visa prover condições mínimas
para que todas as estratégias tenham uma ponto de partida viável.
O principal exemplo de condição necessária é a pré-existência de uma fronteira de decisão,
que só é possível de ser estimada com a presença de alguns rótulos.
Essa escolha também é compatível com aplicações reais, pois,
sendo uma atividade dispendiosa, a rotulação requer cautela.
Por isso, é razoável a obtenção de alguns rótulos antes de se sujeitar aos custos
da amostragem ativa.
Uma maneira de se obter um rótulo por classe é por meio do
 \ing{aprendizado guiado}{guided learning} \citep{conf/kdd/AttenbergP10}.
 Em diversas aplicações, como a categorização de notícias, o usuário é capaz
 de encontrar um exemplar para uma dada classe, seja por páginas de busca
 na \textit{internet} ou por recordação de uma experiência pessoal.
 Assim, o início da amostragem com um exemplo de cada classe pode ser
 feito de forma eficiente em aplicações reais.
 Essa prática tem sido adotada da mesma forma em outros trabalhos
 \citep{conf/nips/GuoS07, conf/cvpr/BiswasP13, conf/iconip/GuJC14a,
 }, com variações mais permissivas com relação à quantidade de exemplos
 \citep{journals/prl/PatraB12} ou substituída por uma amostragem aleatória
 aplicada até que se encontre um exemplo por classe \citep{chermanaprendizado}.

 \ano{settles diz (no livro?) que o critério de parada normalmente é financeiro}
 
Outra característica do cenário está no desconhecimento do orçamento disponível por parte das estratégias implementadas.
Isso implica na geração de sequências de consultas não necessariamente ótimas.
Essa possibilidade, mais teórica do que prática,
decorre de o fato do processo ativo de amostragem não ser uma
\textit{cadeia de Markov} de primeira ordem \citep{conf/icml/RoyM01}.
Uma consulta ótima dependeria de quantas ainda podem ser feitas.

\section{Descrição das bases}
As bases de dados têm suas características expostas no Apêndice \ref{detalhes}
(tabelas \ref{tab:datasetsa},
\ref{tab:datasetsb} e \ref{tab:datasetsc}).
Para maior clareza do tipo de diversidade das bases, alguns grupos especiais foram
criados da seguinte forma:
\begin{itemize}
 \item bases com mais de quatro vezes mais exemplos na classe
majoritária do que na minoritária (Tabela \ref{tab:imb})
 \item bases com mais de cinquenta atributos (Tabela \ref{tab:x})
 \item bases com mais de cinco classes (Tabela \ref{tab:y})
 \item bases com mais de mil exemplos (Tabela \ref{tab:n})
 \item bases com menos de quatrocentos exemplos (Tabela \ref{tab:nm})
\end{itemize}

\input dataset-tables-reduxes

É possível notar a presença de atributos nominais, por exemplo, na Tabela \ref{tab:datasetsa}.
Os algoritmos de aprendizado providos pela ferramenta
Weka \citep{journals/sigkdd/HallFHPRW09}
fizeram eventual uso de seus próprios filtros internos para lidar com esse tipo de atributo,
mas as \elms e as estratégias baseadas na métrica de distância de Mahalanobis precisaram ser
transformadas com a substituição de atributos nominais por numéricos e pela aplicação do
\ing{filtro de padronização}{z-score} \citep{kreyszig2007advanced}
para possibilitar os cálculos e para evitar matrizes mal condicionadas - respectivamente.

\ano{exemplos repetidos foram removidos,
mas mantido um deles com a moda dos rótulos}

\ano{A quantidade de atributos da base Arcene foi reduzida para 998 devido
a restrições de armazenamento do sistema gerenciador de banco de dados}

\tar{
Experiments with Random Projection 2000
% http://cseweb.ucsd.edu/~dasgupta/papers/randomf.pdf
preferem RP(O(n)) a PCA (O(n3));
é uma boa justificativa para ELM, e bom para reduzir
rapidamente as dimensões de bases enormes para o limite de 1000 do mysql.
 Randomness is now a standard tool in algorithm design;
 for instance, it is the basis of the only known polynomial-time
 algorithm for primality testing.
;;;;;
if the projected dimension is high enough, then a
PCA-projected mixture could easily be far better separated
than its randomly projected counterpart. For this reason
PCA remains an important tool in the study of Gaussian
clusters.
}

\section{Algoritmos de aprendizado}\label{learners}
Cada algoritmo de aprendizado tem, intrinsecamente, um viés próprio \ano{REF}.
É natural esperar que esse viés influencie no desempenho das estratégias gnósticas
ou que o aprendizado seja afetado pela sequência de consultas gerada por uma
determinada estratégia agnóstica.
Dessa forma, diferentes algoritmos de aprendizado
precisam ser adotados quando alguma comparação entre estratégias é pretendida.

Seis algoritmos comumente usados e implementados na biblioteca Weka foram adotados
inicialmente:
5-NN\footnote{Cinco vizinhos mais próximos (nome Weka: IBk).},
C4.5w\footnote{Implementação baseada no algoritmo de árvore de decisão C4.5 (nome Weka: J48).},
NB\footnote{\textit{Naive Bayes}},
RF\footnote{\textit{Random Forest}},
SVM\footnote{\textit{Support Vector Machines}},
VFDT\footnote{\textit{Very Fast Decision Trees} ou \textit{Hoeffding Trees}}
\citep{journals/tit/Hart68,
books/mk/Quinlan93,
conf/ecml/Lewis98,
journals/ml/Breiman01,
hearst1998support,
conf/kdd/DomingosH00}.
Adicionalmente, por sua ausência na literatura de aprendizado ativo e características
distintivas com relação aos outros algoritmos,
duas variantes das \elms também foram adotadas.
\ano{duas?}

\subsection{5-NN}
ponderado pelo complemento da distância euclidiana padronizada \ano{REF pra padronização}
busca linear ou em KDTree quando $\frac{|\mathcal{U}|}{|A|}\geq10$

\subsection{C4.5w}
ramificação múltipla (não apenas binária)
fator de confiança na poda $0,25$ (smaller values incur more pruning)
mínimo de $\min{10, 2|Y|}$ exemplos por folhas
poda por substituição pelo nó filho
estimativa de probabilidades com suavização aditiva \cite{books/daglib/0021593}

\subsection{NB}
discretização supervisionada baseada na distribuição normal

\subsection{RFw}
sem poda
escolha aleatória de $\log_2|A| + 1$ atributos por árvore
total de dez árvores

\subsection{SVM}
tipo C-SVC \citep{journals/ml/CortesV95}
núcleo RBF
$\gamma=0,5$ %inverse of the radius of influence of support vectors
% classif/learner g=0.5; SVMmulti g=1

$C=1$ %A low C makes the decision surface smooth, high C aims at classifying all training examples correctly
cache 200MB
eps=0,001 %tolerância do critério de parada SVMmulti eps=0,1
Shrinking(false)
ProbabilityEstimates(true)

\ano{explicar detalhes da implementação svmlib/weka para multiclasse?}

\subsection{VFDT}
Valores padrão exceto  graceperiod
% GracePeriod(math.min(200, pattern.nclasses + newModel.N / 20)
% Gini
% erro permitido de 1e-7
% limiar de desempate de 0,05
%  Minimum fraction of weight required down at least two branches for info gain splitting  (default = 0.01)
%  The number of instances (weight) a leaf should observe before allowing naive Bayes to make predictions (NB or NB adaptive only)(default = 0)

\subsection{CIELM}
nó sigmoide aditivo
\ano{explicar nr de neurônios}

\ano{http://www.sciencedirect.com/science/article/pii/S0925231207000677
mostra tabela comparando tempos onde sigmoide additive é bem mais rápida que RBF.
É uma boa justificativa para se escolher additive nodes.
mostra tb que CI-elm é melhor que I-elm, BP (e RAN e MRAN) em tempo e acc.
}

\section{Estratégias}
Algumas estratégias exigiram configurações específicas ou precisaram de adaptações para
problemas multiclasse.
Essas escolhas são detalhadas nesta seção.

\subsection{SVMsim e SVMbal}
Os códigos originais dos autores das seguintes estratégias \textit{balancedEE}
% journals/jmlr/BaramEL04
foram adaptados para contemplar problemas multiclasse por meio da aplicação de
\ing{rodízio}{round-robin} entre sub-estratégias.
A cada sub-estratégia foi atribuída uma classe principal,
na configuração \textit{um para muitos}.

\subsection{QBC}
\ano{explicar aqui detalhes do RF adotado pra QBC?}

\subsection{Unc e Mar}
Para contemplar problemas multiclasse,
a amostragem por margem (Seção \ref{mar}) foi utilizada como substituta
nos experimentos deste trabalho.

\subsection{EERent e EERacu}
Neste trabalho, seguindo a escolha do artigo original,
optou-se pela entropia como função objetivo.
Adicionalmente, a acurácia balanceada foi adotada como uma variação do método,
visando agir diretamente nas medidas multiclasse de interesse apresentadas na
Seção \ref{metricas}.
Trata-se de uma métrica multiclasse com menos propensão a ter valores próximos
ou iguais a zero do que a medida kappa, ampliando o alcance da medida de
informatividade (Seção \ref{metricas}).
\ano{explicitar essa dedução na Seção \ref{metricas}}
Apenas cem exemplos foram amostrados de $\mathcal{U}$ em cada iteração devido ao
alto custo computacional da estratégia.

\subsection{TU}
\ano{explicar como empreguei mahalanobis nas TUs}

\subsection{Clu}
A implementação original do autor foi adotada neste trabalho com o mesmo
algoritmo de agrupamento chamado
\textit{Ward's average linkage method}\footnote{
Implementação disponível no Weka \cite{journals/sigkdd/HallFHPRW09}.}
\ano{como traduzir?}

\section{Experimento preliminar}\label{preliminar}
Dado o grande número de estratégias, bases de dados e algoritmos de aprendizado envolvidos,
um experimento reduzido \citep{conf/hais/SantosC14} foi conduzido visando
a construção de um panorama geral do comportamento das estratégias
e a definição do dimensionamento mais adequado dos experimentos definitivos
aos recursos computacionais.
Apenas dez estratégias e os algoritmos mais rápidos (C4.5, NB, VFDT e 5-NN)
foram adotados.
Dez execuções de validação cruzada em dez partições totalizaram cem \pools
para cada base - conforme sugerido na literatura de classificação
\cite{conf/pakdd/BouckaertF04}.
As \pools foram consultados até o ponto em que a acurácia máxima da
melhor estratégia houvesse sido atingida.

Aproximadamente um mês\footnote{Pausas devido a problemas técnicos incluídas.}
de experimento foi necessário para apenas a conclusão
nas vinte e oito menores bases de dados.
Assim, considerando-se o crescimento exponencial do custo computacional em alguns
pares \textit{algoritmo de aprendizado}-\textit{estratégia} com o aumento do número
de classes, atributos e/ou exemplos, foi necessário reduzir o procedimento
de validação (Seção \ref{validacao}) e antecipar o término das
consultas.

\section{Avaliação}\label{avaliacao}
\ano{pro final: quando perde, perde muito feio?}

\subsection{Métricas}\label{metricas}
\esb{Um cara que usa log no eixo X e usa ``Area under the Learning Curve'': url.
% http://perso.rd.francetelecom.fr/lemaire/publis/ijcnn\_2\_2011\_camera\_ready.pdf
}

An Easy to Use Repository for Comparing and Improving Machine Learning Algorithm Usage
% http://arxiv.org/pdf/1405.7292.pdf
fala sobre guardar resultados a 'instance level' com as predições do classificadores


\ano{talvez o primeiro artigo que fala em "area under the learning curve" (chamando confusamente de AUClog): When will feature feedback help? quantifying the complexity of classification problems (2007)
O próximo, sem siglas, é do Settles (com f-measure e sem log(x)):
An analysis of active learning strategies for sequence labeling tasks (2008)
No "active chalenge 2011",
que alguns adotam como referência para citar ALC, foi usada a AUC verdadeira e log2(x) no cálculo (igual ao Cawley):
http://www.causality.inf.ethz.ch/activelearning.php?page=evaluation
Porém, eles avaliam sem validação cruzada, talvez por limitação do próprio fato de ser uma competição.
O dataset inteiro é usado uma só vez e o teste é feito em cima de
"all the samples with unknown labels".
Isso explica o fato de classificadores semisupervisionados terem alavancado as estratégias na competição.
Já estou rodando as estratégias agnósticas com 5x5-fold.
Podemos adotar ALC c/ AUC e log(x) para sumarizar os resultados.
}

 Uma medida comum na área de aprendizado ativo
 é a \textit{área debaixo da curva de aprendizado} (Area under the Learning Curve - ALC)
 \citep{journals/jmlr/GuyonCDL11}.
 Ela consiste no somatório de alguma medida de desempenho ao longo
 das consultas.
 Seu intuito é avaliar o desempenho geral de uma estratégia, ou seja,
 numa ampla faixa de orçamentos.
 Assim, a medida se mostra adequada para a tarefa de comparação
 de estratégias.
 Ela foi adotada em conjunto com a medida kappa -
 resultando no valor médio da medida para toda a sequência de consultas.

A medida kappa de Cohen \citep{journals/coling/EugenioG04}
é utilizada para comparar o grau de consenso entre avaliadores
tem demonstrado bom poder discriminatório frente a outras medidas
que sumarizam a matriz de confusão \citep{conf/acivs/DemirkesenC08}.
Quando aplicada como medida de desempenho de classificação,
ela assume valor $1$ para acerto total, $0$ para desempenho equivalente ao
aleatório e $-1$ para erro total.
Valores intermediários indicam o quão bom ou ruim é o clasificador.
\cite{conf/pkdd/Shah11} apresenta a versão multiclasse para a medida kappa
que foi adotada neste trabalho.
Ela tem a importante propriedade de remeter à medida original em problemas binários.
Por fim, ela permite realizar a comparação por dois pontos de vista:
interestratégias e em relação ao acaso.
\ano{colocar fórmula da kappa? já incluir pseudoALC na fórmula? explicar depois da formula que
a ALC pro kappa se torna o kappa médio}

Outro ponto de vista importante é em relação ao desempenho passivo.
Sendo ele dado por apenas um valor para todos os exemplos, não existe
a possibilidade de se calcular uma ALC.
Assim, deve ser definido um momento na curva de aprendizado como o
limite do orçamento disponível e deve ser avaliado o modelo induzido até então.
\green{baseline p/ AccBal é $1/|Y|$ ?}
O orçamento escolhido foi $\cent=\frac{|\mathcal{U}|}{2}$
% metade dos exemplos disponíveis
nas trinta e nove bases com menos de quatrocentos exemplos (Tabela \ref{tab:nm})
e  $\cent=200$ nas bases restantes.
% duzentos exemplos

Outra medida possível é a acurácia balanceada \citep{journals/bmcbi/MassoV10}.
Assim como a kappa multiclasse, ela é adequada para problemas multiclasse desbalanceados.

\ano{colocar fórmula da accbal?}



\subsection{Validação}\label{validacao}
Nos experimentos,
para viabilidade computacional e conforme literatura de aprendizado ativo \ano{REF}
foram adotadas cinco execuções de validação cruzada em cinco partes -
totalizando vinte e cinco \pools para cada base.

\tar{mencionar detalhes de embaralhamento? e seeds de classificador?}

\ano{A survey of cross-validation procedures for model selection (2010)
{http://www.di.ens.fr/willow/pdfs/2010\_Arlot\_Celisse\_SS.pdf}
;;;
tem até review de fast CVs.
;;;
According to Breiman and Spector (1992) the best risk estimator is LOO,
whereas 10-fold CV is more accurate for model selection.
}

\subsection{Teste estatístico}
\ano{citar programa do Covões}

explicar critério de empate não arbitrário que inventei, pois na construção da árvore de nichos
haveria confusão se adotasse apenas os vencedores, por causa da existência de estratégias similares.


%%%%%%%%%%%%%%%%%%%%%%%%%%%%%%%%%%%%%%%%%%%%%%%%%%%%%%%%%%%%%%%%%%%%
\section{Ferramentas}\label{sec:ferramentas}
Foram utilizados um total de 188 núcleos\footnote{\textit{Núcleos} por simplicidade,
mas na realidade cada computador oferecia diferentes unidades de processamento;
sendo o outro tipo as \textit{threads} na nomenclatura da Intel.} de processamento:
\begin{itemize}
\item 24 núcleos de 2,4GHz da \textit{nuvem} USP;
\item 108 núcleos de 3,5GHz do \textit{cluster} Biocom;
\item 24 núcleos de 2,5GHz do servidor GPU do laboratório Biocom;
\item 8 núcleos de 4GHz da estação de trabalho; e
\item 8 núcleos de 2,4GHz do \textit{notebook} de trabalho\footnote{Ambos,
estação e \textit{notebook} vinculados ao orientador do autor deste trabalho.}.
\end{itemize}

\green{descrever hardware, mysql}

scala

pgfplots

java

mtj

apache commons math

%software/algorithms/hardware
The library Matrix-Toolkits-Java was used to  interface the reference implementation of LAPACK with Debian default compilation.
%reference implementation of LAPACK para evitar instabilidade numerica (= matriz singular ocorrer muito cedo)
Intel(R) Core(TM) i7-3630QM CPU @ 2.40GHz