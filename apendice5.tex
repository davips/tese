\chapter{Revisão sistemática}\label{revsist}

\ano{a revisão sist. não chegou a lugar algum, não compensaria dedicar um capítulo inteiro a
ela; porém no final de tudo, com a tese quase pronta, posso readequá-la e incluí-la}


\ano{tema: active learning
problema: há várias estratégias, como escolher uma delas?
protocolo de revisão:
        background: framework teórico; problemática e objetivo
        questões de pesquisa:
        estratégia de busca: palavras-chave, sinônimos e relacionadas
        critérios: tipo de estudo, de inclusão e exclusão
        seleção: etapas, plano
        avaliação da qualidade: critérios
        extração de dados: como, o quê
        calendário: }


        \blue{Revisão bibliográfica sistemática é um procedimento de identificação,
avaliação e interpretação de toda pesquisa relevante disponível a respeito
de uma particular questão ou assunto de interesse \citep{Kitchenham2007}.
O intento original da revisão sistemática era suportar a medicina baseada em evidência.
Atualmente, trata-se de um método também bastante utilizado na área de engenharia de
\textit{software}.
Por outro lado, na área de aprendizado de máquina,
esse tipo de revisão é uma tendência mais recente
e com poucas publicações como é o caso do trabalho de \cite{journals/jbcs/PisaniL13}.
Na base de dados Scopus\footnote{\textit{http://www.scopus.com} - 02/06/2014}, por exemplo,
há apenas 16 trabalhos na área sobre ou que apliquem revisão sistemática
enquanto que, para engenharia de \textit{software}, há 203.
}

% Neste capítulo são apresentados os ?? e resultados da revisão bibliográfica.
%
% questões que me vieram à mente trabalhando na área:
% Que garantias existem de que uma estratégia de aprendizado ativo é economicamente viável?
% \# Essa questão me interessa porque ela põe em cheque a real utilidade de AL.
%    Elas são significativamente melhores do que a amostragem aleatória?
%    Em que situações o desempenho é pior?
%    Quando a variabilidade no desempenho se torna um risco para o orçamento?
% \# Há teoria de algumas estratégias no que diz respeito a se ter garantias de não ser pior do que
% aleatório.
% \# A melhor maneira de averiguar esses pontos é por meio de experimentação.
% \# Trata-se de um problema extremamente relevante.
%
% As estratégias da literatura são usadas na prática?
% Há algoritmos de aprendizado mais adequados do que outros para aprendizado ativo?
% É possível ter uma estratégia boa para qualquer algoritmo de aprendizado? Senão, como escolher?
% É possível ter uma estratégia boa para qualquer domínio de aplicação? Senão, como escolher?
% As estratégias existentes são viáveis para interação com o usuário em tempo real?
% Quais são as estratégias de aprendizado ativo mais relevantes?
% Quais são os principais paradigmas em que as estratégias se agrupam?
%
% background
%
% propósito/objs - apresentar evidencias
%
% inc/exc criterios
%
% search strat
%
% how will be setlected
%
% how assesssd
%
% como extrair dados
%
%
% titulo, onj e questao devem dizer a mesma coisa
%
%
% sinonimos - existem termos equivalentes, mas pertencentes a outras areas.
%
%
% temas recorrentes: desbalanceamento
% \section{Metodologia}
% % coisas comuns às três revisões
%
% Visando artigos recentes, com alguma garantia de qualidade
% Scopus-Ciências físicas (7.200 publicações)- base de acesso restrito, foram selecionados apenas
% artigos dos últimos quatro anos (140/141), devido ao grande número de resultados (944)
% ACM (411.762 publicações próprias e de afiliados) - base com poucos resultados (57/57)
% Springer-Computação (3.321) ALcomAcessoAoPDF(21)
% Visando artigos influentes, ou seja, com muitas citações.
% Artigos muito recentes ficam naturalmente excluídos por tenderem a ter menos citações -
% exceto artigos recentes de alto impacto.
% Google Acadêmico-Computação (5.350.000 publicações) - (19700) Foram selecionados apenas os
% artigos com mais de 25 citações na ordem de relevância determinada pelo critério interno do Google
% (121/130)
% IEEE (3.748.954 publicações) - (382) Foram selecionados apenas os 48/50 artigos mais citados, o
% que equivale a artigos com mais de quatro citações
%
% 10 duplicados foram encontrados automaticamente
%
% A literatura de aprendizado ativo normalmente deixa o cenário baseado em pool implícito.
% Por isso, é necessário excluir os outros cenários na consulta.
%
% \section{Aprendizado Ativo}
% Não é do conhecimento do autor nenhuma revisão sistemática sobre aprendizado ativo,
% porém existem revisões convencionais sobre os diferentes cenários e estratégias.
% Tais revisões são usadas para complementar a revisão sistemática ??.
%
% Primário: fazer uma prospecção da área em busca das estratégias existentes para o
%
%
% % mapeamento sistemáticos empregados nesta tese.
% % Esses procedimentos visam a replicabilidade e não-enviesamento só isso?? da
% % revisão bibliográfica e foram empregados conforme as recomendações
% % encontradas na literatura específica de aprendizado de máquina \cite{}.
% %
% % O mapeamento foi necessário para explorar a vasta área
% % de aprendizado ativo de maneira eficiente em busca
% % das questões mais importantes,
% %
% %
% % \subsection{Planejamento}
% % questões
% % protocolo (definição da informação-alvo, estratégias de busca, seleção e avaliação das
% % referências)
% % \subsection{Condução}
% % maximizar refs que respondam às perguntas
% % estudos primários (
% % excluir irrelevantes, checar qualidade, extrair info,
% % sumarizar
% % )
% % estudos secundários
% %
% % \subsection{Resultados}
% %
% %
% %
% %
% %
% %
% % AL
% % Vantagens/razão de ser?
% % Qual o estado atual da área? (desenvolvimento teórico Hanneke/Dasgupta/Balcan/Beygelzimer e
% propostas statist. sólidas; Settles, duo lingo, musica e propostas ad hoc)
% % Estratégias?
% % Métricas de performance? (ALC, )
% % Formas de avaliação? (pool=tr+ts, 5-fold, 50/50 holdout)
% % Datasets? Há como comparar resultados?
% %
% % Questão principal:
% % Quando AL é efetivo?
% % Quais vantagens de se adotar AL? E desvantagens?
% %
% % definir palavras-chave (consultas)
% %
% %
% %  ACM Digital Library
% % (http://dl.acm.org/)
% % ? IEEE Xplore
% % (http://ieeexplore.ieee.org/)
% % ? Science Direct
% % (http://www.sciencedirect.com/)
% % ? Web of Science
% % (http://isiknowledge.com/)
% % ? Scopus
% % (http://www.scopus.com/)
% %
% %
% % Critério de exclusão:
% % . Publications that do not deal with keystroke dynamics
% % for intrusion detection: the aim of this review is to work
% % with intrusion detection, which comprehends authentica-
% % tion systems. Therefore, references that do not meet this
% % requirement were not included.
% % 2. Publications with one page, posters, presentations, abstra-
% % cts and editorials, texts in magazines/newspaper and
% % duplicate publications in terms of results, except the most
% % complete version: references without enough informa-
% % tion to answer the research question. This criterion also
% % avoids unnecessary work for the cases in which the same
% % study is published in different versions.
% % 3. Publication hosted in services with restricted access and
% % not accessible or publications not written in English.
% %
% % Avaliação da qualidade
% %
% % design, conduction, analysis and conclusion:
% %  Kitchenham B, Charters S (2007) Guidelines for performing sys-
% % tematic literature reviews in software engineering, technical report
% % 2007?001. Keele University and Durham University Joint Report
% % 1. Were the goals clearly presented in the beginning of the
% % work?
% % 2. Were the advantages/disadvantages of keystroke dynam-
% % ics discussed?
% %
% % 3. Is the dataset available to be reused?
% % 4. Was it detailed how the feature vector is generated?
% % 5. Were the values of the algorithm parameters presented?
% % 6. Were the applied approaches detailed so as to allow them
% % to be replicated?
% %
% % 7. Were experimental tests conducted?
% % 8. Were the results compared to previous researches in the
% % area?
% %
% % 9. Were the limitations of the study presented?
% %
% % Extrair dos selecionados
% % ? Basic information about the publication (title, authors,
% % name and year of publication)
% % ? Were performance tests conducted?
% % ? Type of device (e.g. PC, mobile)
% % ? Best performance achieved: algorithm, measure and
% % performance
% % ? Number of users in the tests
% % ? Algorithms used in the tests
% % ? Extracted features
% % ? Is the test dataset available to be reused? Where?
% % ? Type of verification: static text or dynamic text?
% % ? Observations
% %
% % organiza deduplica etc
% %  http://www.mendeley.com/
% %
% %  tabela com totais  por database
% %
% %
% %  only those papers with quality score
% % equals or higher than 7.5 were considered, resulting in 16
% % publications.
% %
% % grafico qtdaXQualty qtadweXano e
% %
% %
% % tabela comparando estrategias
% %
% % histograma de adoção de metodos ou de ter uma propriedade especifica
% %
% %  tabela de classificadores adotados
% %
% %
% %
% %
% %
% %  ==============================
% % reviews of AL
% % combining/meta AL
% % exp. evaluations for AL
% % ELM+AL
% % ELM fast/interactive/growing/incremental
