\section{Terminologia e nota��o}\label{notacao}
O conjunto de s�mbolos e termos empregados neste texto s�o definidos nesta se��o.
Ele � baseado no livro de \cite{series/synthesis/2012Settles} juntamente com a nota��o
normalmente utilizada na literatura de
\elms e meta-aprendizado \citep{journals/tsmc/HuangZDZ12,books/daglib/0022052}.

O termo \textit{algoritmo de aprendizado} � substitu�do por \textit{classificador}
quando se trata do produto do processo, isto �, o modelo capaz de classificar exemplos.
Durante a indu��o do modelo, se ele integrar uma estrat�gia de aprendizado ativo,
ele � chamado de \textit{aprendiz}.
Apesar de contradizer o princ�pio motivador do aprendizado ativo explicitado
na Se��o \ref{aprendizado-ativo},
existem estrat�gias sem aprendiz - chamadas \textit{agn�sticas}.
Cada \textit{consulta} de uma dada estrat�gia ao \textit{or�culo} visa a obten��o
do \textit{r�tulo} que indica a \textit{classe} $\bm{y}$ de um dado exemplo dentre um
conjunto limitado $Y$ delas. Assim, o n�mero de classes � indicado por $|Y|$.

As consultas podem ser \ing{explorat�rias}{for exploration}, que buscam maximizar
a variedade na escolha de exemplos; ou \ing{prospectivas}{for exploitation},
que se concentram apenas nos casos mais cr�ticos,
ou seja, com maior \textit{informatividade}.
O limite de dura��o para processamento computacional entre consultas
� chamado \ing{tempo de espera toler�vel}{tolerable waiting time},
pois � o tempo efetivamente sentido pelo or�culo.

\esb{$A(X)$ � o conjunto de atributos.
Cada atributo $\bm{a} \in A$ � o conjunto de todos os valores que cada tupla
$\bm{x}\in X$ pode assumir.}


\esb{ e a quantidade inicial de exemplos �
$|\mathcal{U}|$ - mantendo a nota��o
usual de os s�mbolos $\mathcal{U}$ e $\mathcal{L}$ representarem a
reserva de exemplos \citep{series/synthesis/2012Settles}
e o conjunto de exemplos j� rotulados, respectivamente.}
$ID$ densidade de informa��o
$TU$ utilidade de treinamento
$Inf(\bm{x})$ medida de informatividade
$d(\bm{x},\bm{u})$ medida de dist�ncia
$sim(\bm{x},\bm{u})$ medida de similaridade
$\cent$ or�amento ou n�mero de consultas

