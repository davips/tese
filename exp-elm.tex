% % Máquinas Extremas de Aprendizado Ativo
% \section{Adaptações de \elms}\label{elmnovas}
% A código das adaptações apresentadas nas próximas seções e outras variantes
% está disponível publicamente \citep{doi/elm}.
% 
% \esb{estudo de caso para um dado learner}
% 
% \tar{menciona inexistência da combinação de ELM e AL na literatura
% e mostra AL sendo melhor que aleat e nao muito pior que passiva;
% explorar resultados separadamente pra cada tipo de ELM}
% 
% \tar{ impossível completar todas as bases, mas ok;
% quanto mais memória, mais datasets}
% 
% \ano{ELM ainda não sabe lidar com exemplos ponderados! (necessário para SGmulti*)}
% 
% \subsection{I-ELM interativa}
% 
% As propostas incrementais para IELM e CIELM mostraram-se
% positivamente impactadas pelo aprendizado ativo no experimento principal.
% % \subsection{EIELM}
% 
% \subsection{CI-ELM interativa}
% 
% \subsection{OSEM-PRESS-ELM}
% \ano{foi usada ELM com topologia raiz(N) no segundo experimento preliminar para demonstrar viabilidade
% de AL com ELM \citep{santos2014viabilidade};
% e tb foi usada como metaclassificador no último experimento}
% \tar{citar apendice}
% 
% \section{Máquina extrema de aprendizado ativo baseada no impacto esperado}
% experimenta ExpELMChange
% (AL especifico pra nintera e/ou pra IELM e/ou CIELM) -
% comparar com outras estrats.