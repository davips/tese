\chapter{M�quinas Extremas de Aprendizado Ativo}\label{ealm}
A c�digo das adapta��es apresentadas nas pr�ximas se��es e outras variantes
est� dispon�vel publicamente \citep{doi/elm}.

\esb{estudo de caso para um dado learner}

\tar{menciona inexist�ncia da combina��o de ELM e AL na literatura
e mostra AL sendo melhor que aleat e nao muito pior que passiva;
explorar resultados separadamente pra cada tipo de ELM}

\tar{ imposs�vel completar todas as bases, mas ok;
quanto mais mem�ria, mais datasets}

\section{IELM}

As propostas incrementais para IELM e CIELM mostraram-se
positivamente impactadas pelo aprendizado ativo no experimento principal.
% \subsection{EIELM}
\section{CIELM}
\section{nintera}
\ano{foi usada ELM com topologia raiz(N) no segundo experimento preliminar para demonstrar viabilidade
de AL com ELM \citep{santos2014viabilidade};
e tb foi usada como metaclassificador no �ltimo experimento}
\tar{citar apendice}

\section{ExpELMChange}
experimenta ExpELMChange
(AL especifico pra nintera e/ou pra CIELM) -
comparar com outras estrats.