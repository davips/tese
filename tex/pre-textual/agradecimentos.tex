Algumas virtudes surgem sem anúncio.
Imagino que a experiência quádrupla da paternidade/maternidade tenha surpreendido meus pais em muitos aspectos.
Hoje compreendo os momentos possivelmente mais difíceis, aqueles em que a paciência e dedicação é resultado de puro esforço - quando é preciso manter o mundo girando, com ou sem instinto, romantismo ou qualquer outro impulso adicional.
Expresso aqui meu agradecimento a Julio e Eliana, e meus irmãos.
Similarmente, a orientação acadêmica também traz seus ensinamentos para a vida.
Desde o prof. João da época do mestrado e seus sábios conselhos existenciais disponíveis até hoje, até o prof. André, exemplo de ânimo e humildade.
Agradeço a ele pela sua permanente disposição em ajudar e sua paciência ao longo de cinco anos.
Agradeço também às contribuições da banca, cujo interesse e dedicação na apreciação do trabalho ajudaram a dar significado a este momento.

Este trabalho mal teria um começo sem a ajuda de Luís P. Garcia e certamente teria sido mais difícil sem o apoio ou material de Angela Giampedro, Lhaís V., T. Covões, R. Mantovani, R. Cerri, P. Pisani, P. Jaskowiak, L. Colleta, D. Horta e F. Zuher.
Outras pessoas, como R. Sadao, R. Rigolin, Rafaela D., Uziel, Ligia T., Danira e Fernando, Virgínia V., Camila T., Elaine K., Pâmela C., Vital F., T. Galante, M. Manzato, Herr Marcos V., Nayara, Renan e Larissa Campos merecem meu carinho por terem sido minha família sancarlense por tantos anos.
Por fim, escrevo em memória das gigantes esverdeadas que me acompanharam desde o ingresso na universidade, em São Paulo, São Carlos e Araraquara, mas acabaram convertidas em serragem.

%\begin{center}
%\copyright Copyright 2016 - Davi Pereira dos Santos
%
%Todos os direitos reservados
%\end{center}
{\footnotesize
Pesquisa desenvolvida com apoio n\textordmasculine DS-3136101/D da CAPES\footnote{Coordenação de Aperfeiçoamento de Pessoal de Nível Superior} no período de 01/04/2011 a 28/02/2015 e utilização dos recursos computacionais do CeMEAI\footnote{Centro de Ciências Matemáticas Aplicadas à Indústria} financiados pela ag. FAPESP\footnote{Fundo de Amparo à Pesquisa do Estado de São Paulo}.
\textit{I would like to thank J. Hammersley for sponsoring the hosting of my thesis} at \url{https://www.overleaf.com}.

Este texto foi escrito com ferramentas disponibilizadas pela comunidade de \textit{software} livre:
\LaTeX, github\footnote{\url{https://github.com}}/git 2.1.4, Debian 8.2, Kile 2.1.3, Okular 4.14.2, PGFPlots 1.12 e outras a elas vinculadas.
}

