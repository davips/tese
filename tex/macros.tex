\DeclareMathAlphabet{\mathcal}{OMS}{cmsy}{m}{n}
\usepackage{pbox}
\usepackage{marvosym}
\usepackage{bm}
\usepackage{wasysym}
\usepackage{makecell}
\usepackage{graphicx}
\usepackage{icomma}
\usepackage{afterpage}
\usepackage{pdflscape}
\usepackage{rotating}
% \usepackage{subcaption}
\usepackage{pgfplots}
\usepackage{amsopn}
\usepackage{soul}

\pgfplotsset{compat=1.10}
\usetikzlibrary{trees}
\usetikzlibrary{patterns}
\usepgfplotslibrary{fillbetween}
\usetikzlibrary{fadings}
\tikzfading[name=myfading, top color=transparent!0, bottom color=transparent!80]
%melhor estilo pra 1a pg do capítulo
\chapterstyle{ell}
% sumario melhor
% \renewcommand{\cftchapterfont}{\sffamily}   
\renewcommand{\cftsectionfont}{\normalfont\sffamily}  
\renewcommand{\cftsubsectionfont}{\itshape\sffamily} 
\settocdepth{section}
% headline melhor pras seções e capítulo
\renewcommand*{\chapnumfont}{\bfseries\HUGE\sffamily}
\renewcommand*{\chaptitlefont}{\normalfont\huge\sffamily}
\setsecheadstyle{\bfseries\Large}
\setsubsecheadstyle{\bfseries\normalsize}

\DeclareMathOperator*{\unidist}{U}
\DeclareMathOperator*{\cov}{cov}
\DeclareMathOperator*{\argmax}{arg\,max}
\DeclareMathOperator*{\argmin}{arg\,min}
\DeclareMathOperator*{\sk}{sk}
\DeclareMathOperator*{\ku}{ku}
\DeclareMathOperator*{\qnc}{\#nc}
\DeclareMathOperator*{\qnom}{\#no}
\DeclareMathOperator*{\qnum}{\#nu}
\DeclareMathOperator*{\qatt}{\#at}
\DeclareMathOperator*{\qexe}{\#ex}
\DeclareMathOperator*{\qexeatt}{\#ea}
\DeclareMathOperator*{\lgqexe}{lgex}
\DeclareMathOperator*{\lgqexeatt}{lgea}
\DeclareMathOperator*{\nom}{isnom}
\DeclareMathOperator*{\pno}{\%no}
\DeclareMathOperator*{\en}{en}
\DeclareMathOperator*{\corr}{cr}
\DeclareMathOperator*{\cn}{cn}
\DeclareMathOperator*{\cnk}{cnk}
\DeclareMathOperator*{\si}{si}
\DeclareMathOperator*{\sik}{sik}
\DeclareMathOperator*{\du}{du}
\DeclareMathOperator*{\duk}{duk}
\DeclareMathOperator*{\dime}{dim}
\DeclareMathOperator*{\info}{Inf_\theta}
\DeclareMathOperator*{\stratID}{ID_\theta}
\DeclareMathOperator*{\stratIDTU}{ID_{TU_\theta}}
\DeclareMathOperator*{\JS}{JS}
\DeclareMathOperator*{\at}{\bm{a}}
\DeclareMathOperator*{\simi}{sim}
\DeclareMathOperator*{\limiar}{limite}
\DeclareMathOperator*{\valor}{-valor}
\DeclareMathOperator*{\limiaring}{max}

\definecolor{dark-green}{rgb}{0,0.5,0}
\newcommand{\parei}[1]{\phantom{vvvvvvvvvvvvvvvvvvvvvvvvvvvvvvvvvvvvvvvvvvvvvvvvvvvvvvvvvvvvvvv}
\ano{--------------- parei aqui: #1 ------------------}
\phantom{vvvvvvvvvvvvvvvvvvvvvvvvvvvvvvvvvvvvvvvvvvvvvvvvvvvvvvvvvvvvvvv}}
\newcommand{\destaque}[1]{\textbf{\textit{#1}}}
\newcommand{\opar}{min5NNw}
\newcommand{\X}{X\xspace}
\newcommand{\Y}{Y\xspace}
\newcommand{\xt}{\bm{x}_{(t)}\xspace}
\newcommand{\x}{\bm{x}\xspace}
\newcommand{\yt}{y_{(t)}\xspace}
\newcommand{\pool}{reserva de exemplos\xspace}
\newcommand{\pools}{reservas de exemplos\xspace}
\newcommand{\Upool}{\mathcal{U}\xspace}
\newcommand{\Ut}{\U_{(t)}\xspace}
\newcommand{\Lt}{\mathcal{L}_{(t)}\xspace}
\newcommand{\tra}[2]{\underline{#1}\xspace\footnote{\textit{#2}\xspace}}
\newcommand{\blue}[1]{\textcolor{blue}{#1}\xspace}
\newcommand{\red}[1]{\textcolor{red}{#1}\xspace}
\newcommand{\green}[1]{\textcolor{dark-green}{#1}\xspace}
\newcommand{\esb}[1]{\blue{#1}\xspace}
\newcommand{\ano}[1]{\red{[$\star$ #1 $\star$]}\xspace}
\newcommand{\tar}[1]{\green{$\star$ #1}\xspace}
\newcommand{\versionspace}{espaço de versões\xspace}
\newcommand{\come}[1]{\green{\footnotesize\phantom{i}$\triangleleft$\phantom{i}\textit{#1}\xspace}}

\newcommand{\ing}[2]{\emph{#1}\footnote{[\textit{#2}]}}
\newcommand{\novo}[1]{\emph{#1}}
% \newcommand{\eer}{redução do erro esperado\xspace}
% \newcommand{\Eer}{Redução do erro esperado\xspace}
\newcommand{\elms}{máquinas de aprendizado extremo\xspace}
\newcommand{\elm}{máquina de aprendizado extremo\xspace}
\newcommand{\svm}{máquina de vetores de suporte\xspace}
\newcommand{\bom}[1]{\textcolor{blue}{\textbf{#1}}\xspace}
\newcommand{\bomd}[1]{\textbf{#1}\xspace}
\newcommand{\ruim}[1]{\textcolor{red}{\textbf{#1}}\xspace}

\newcommand{\tarefa}[1] {\addcontentsline{toc}{section}{\tar{#1}}}

\hyphenation{su-per-vi-sio-na-dos}
\hyphenation{su-per-vi-sio-na-do}
\DeclareMathOperator*{\stratIDATU}{ID_{ATU}}



\pgfplotsset{marst/.style={solid, violet, mark=*, mark options={solid, scale=1.4}, mark repeat=60}}
\pgfplotsset{sgst/.style={solid, thick, orange, mark=x, mark options={solid, scale=2}, mark repeat=60}}
\pgfplotsset{eerst/.style={solid, red, mark=square*, mark options={solid, scale=1.4}, mark repeat=60}}
\pgfplotsset{hsst/.style={dashed, black, mark=triangle*, mark options={solid, scale=1.4}, mark repeat=60}}
\pgfplotsset{tust/.style={solid, gray, mark=square, mark options={solid, scale=1.4}, mark repeat=60]}}
\pgfplotsset{rndst/.style={loosely dashed, olive, mark=triangle, mark options={solid, scale=1.4}, mark repeat=60}}
\pgfplotsset{atust/.style={dotted, mark=star, ultra thick, blue,  mark repeat=50, mark options={solid, scale=1.9}, mark repeat=60}}
\pgfplotsset{htust/.style={solid, thick, green!50!black, mark=10-pointed star, mark options={solid, scale=1.9}, mark repeat=60}}
\pgfplotsset{dwst/.style={solid, thick, gray, mark=diamond, mark options={solid, scale=1.4}, mark repeat=60}}
\pgfplotsset{eer2st/.style={solid, brown, mark=pentagon, mark options={solid, thick, scale=1.4}, mark repeat=60}}
\pgfplotsset{htu2st/.style={dotted, blue, mark=triangle, mark options={solid, scale=1.5}, mark repeat=60}}
\pgfplotsset{defst/.style={loosely dashed, thick, blue, no marks}}

\usepackage{alltt}
\usepackage{xcolor}
\usepackage{lmodern}