As propostas deste trabalho foram avaliadas empiricamente por meio de experimentos comparativos baseados em estratégias de aprendizado ativo e valores de referência da literatura.
A coleção consistiu de noventa conjuntos de dados (Capítulo \ref{metodologia}).
A apresentação dos resultados é dividida em duas partes principais que correspondem às seções \ref{expbase}, \ref{expmeta}, respectivamente:
\begin{description}
   \item [nível base,] em que as estratégias propostas e o efeito do meta-aprendiz utilizando PCT (\textit{Predictive Clustering Trees}) são avaliados - incluindo a investigação de algumas relações entre estratégias, algoritmos de aprendizado e conjuntos de dados; e,
   \item [nível meta,] que contém evidências da ocorrência de aprendizado na proposta de recomendação automática, tanto com o algoritmo PCT adotado inicialmente, quanto com outros algoritmos de aprendizado utilizados posteriormente - incluindo uma breve análise dos meta-atributos mais relevantes.
\end{description}
Adicionalmente, a Seção \ref{outmod} reporta os resultados da extensão da análise no nível meta a outras modalidades de recomendação.

Considerações gerais são feitas na Seção \ref{sintese}.
Detalhes sobre os aspectos básicos das técnicas de avaliação envolvidas foram previamente descritos no Capítulo \ref{metodologia}.
Um breve experimento com um novo conjunto de algoritmos de aprendizado, baseados em comitês, será apresentado, separadamente, no Apêndice \ref{apexpcom}.

\section{Nível base}\label{expbase}
\input tex/exp-analise

\section{Nível meta - Recomendação de algoritmos}\label{expmeta}
\input tex/exp-meta

% \newpage
\section{Nível meta - Outras modalidades}\label{outmod}
\input tex/exp-meta2

% \newpage
\section{Considerações}\label{sintese}
\input consideraexps