Na literatura de aprendizado ativo, muitas abordagens têm sido propostas para a realização de consultas relevantes junto ao oráculo.
O principal aspecto dessa área, explorado neste trabalho, é a pouca atenção dada  
à relevância dos algoritmos de aprendizado enquanto aprendizes, especialmente se consideradas as especificidades de cada estratégia de amostragem e cada conjunto de dados.
Foi necessária, portanto, uma investigação dos fatores que influenciam o \textit{viés de aprendizado ativo}, tais como a presença e tipo do \textit{viés de aprendizado} do algoritmo e o \textit{viés de amostragem} da estratégia.

A investigação conduzida resultou nesta tese, que mostra ser possível atingir melhores desempenhos preditivos de acordo com as propriedades do conjunto de dados e o momento em que se situe o processo de aprendizado.
O meio proposto é uma escolha (ou inibição) mais criteriosa,
% manual, se feita por meio da árvore impressa nos experimentos; semiautomática, se feita por meio da recomendação de ranqueamento; ou automática, se feita por meio da recomendação do melhor algoritmo.
preferencialmente automática, do tipo do viés de aprendizado, da estratégia ou de ambos.

O desenvolvimento da pesquisa requereu trabalho conceitual, de criação, de implementação e metodológico.
O autor deixa neste parágrafo registrada a extensão do esforço empreendido no desenvolvimento e a carga computacional resultante dos experimentos.
Embora os experimentos realizados nesta pesquisa tenham tido um elevado custo computacional, sua real finalidade é a economia em termos de esforço humano, provavelmente mais custoso que a energia consumida.
Nesse aspecto, este trabalho pode ser considerado um projeto bem sucedido, no mínimo, pela confirmação da efetividade do aprendizado ativo em geral.
Além disso, a pesquisa também resulta em benefícios intangíveis, que estão além da mera aplicabilidade prática de seus resultados. Durante a pesquisa, 13 estratégias e variações foram estudadas, implementadas e propostas, totalizando 22 alternativas, se considerados os parâmetros explorados.
Uma coleção com 90 conjuntos de dados foi elaborada visando, tanto quanto possível, um embasamento estatístico confiável para as observações feitas.
Foram realizados experimentos no nível base, aprendizado de máquina convencional, e no nível meta, para o meta-aprendizado.
Quatro algoritmos de aprendizado, com vieses e desempenhos preditivos variados,
%demonstrados como distintos, 
e dois comitês foram empregados no nível base.
No nível meta, duas medidas de referência e quatro comitês foram utilizados.
O esforço de implementação e custo computacional podem ser resumidos pelo número de 40 milhões de consultas registradas; rotuladas por um incansável oráculo, que felizmente era simulado - mais detalhes são dados no Apêndice \ref{apfer}. Numa aplicação real, essa tarefa seria atribuída a um especialista no domínio correspondente.

Por fim, esta pesquisa enfrentou algumas dificuldades, listadas na Seção \ref{difi}, que não impediram o atingimento das metas.
As contribuições, cristalizadas pela comprovação das hipóteses formuladas, são apresentadas nas seções \ref{metas} e \ref{conchipoteses}, respectivamente.
Adicionalmente, algumas limitações e possibilidades de trabalhos futuros foram identificadas e apresentadas nas seções \ref{limitacoes} e \ref{futuro}, respectivamente.

\section{Dificuldades}\label{difi}
Embora essencial para aumentar a generalidade dos resultados, o tamanho da coleção organizada criou uma dificuldade sem relatos prévios na pesquisa bibliográfica realizada.
Normalmente, as estratégias são testadas em poucos conjuntos (Seção \ref{motiv}) frequentemente acompanhados da ilustração individual de suas respectivas curvas de aprendizado.
Durante a pesquisa que resultou no presente trabalho, houve diversas tentativas de sumarização de todas as curvas no mesmo gráfico, como: normalização de medidas ou mesmo de compatibilização entre diferentes limites de orçamento; contagem de vitórias; e, neutralização das especificidades de cada conjunto de dados por meio da subtração do desempenho de estratégias de referência (Rnd, limites teóricos superior e inferior etc.) - entre outras. 
Finalmente, as curvas de ranqueamento, propostas na Seção \ref{sec:curvas}, se mostraram a solução mais direta do ponto de vista conceitual, desde que todos os conjuntos de dados fossem consultados com o mesmo limite de orçamento.
Uma limitação desse tipo de curva, de certa forma positiva, é requerer que muitos conjuntos sejam adotados - o oposto da limitação das curvas convencionais, para as quais apenas poucos conjuntos podem ser viavelmente representados.

Há uma tendência ao emprego da ALC na literatura (\textit{Area Under the Learning Curve} - Seção \ref{medidas}).
Contudo, não há consenso sobre a medida que a deva compor.
Esse problema também ocorre em outras áreas que envolvam a tarefa de classificação.
Optou-se pelo índice $\kappa$ (kappa multiclasse - Seção \ref{newhtu}) devido à sua interpretabilidade e à sua dependência de poucos dados, permitindo que apenas a matriz de confusão fosse armazenada.
Os dados necessários para o cálculo de medidas como a \sigla{AUC}{\textit{Area Under the ROC Curve}} \cite{lobo2008auc} fariam com que a base de dados excedesse o espaço disponível em disco.
A alternativa, que seria registrar o valor da AUC e descartar dos demais dados, aumentaria o custo computacional, não seria adequada para todos os algoritmos adotados e não permitiria a troca para outras medidas de desempenho posteriormente.

O cálculo da distância de Mahalanobis mostrou-se proibitivamente custoso, fazendo com que essa métrica fosse adotada apenas no experimento de recomendação de métricas de distância (Seção \ref{recdist}).
Provavelmente, o uso de toda a reserva de exemplos para seu cálculo não tenha sido uma boa escolha.

Por fim, as bibliotecas de estratégias implementadas por terceiros se mostraram muito modestas, com poucas, ou pouco relevantes, abordagens implementadas.
Essas bibliotecas compreendem um grupo heterogêneo de linguagens de programação e de cenários de aprendizado ativo.
Alguns exemplos encontram-se na página de Burr Settles na \textit{internet}\footnote{\url{http://active-learning.net} - Acessado em 07/01/2016.}.
A única implementação de terceiros efetivamente utilizada nos experimentos foi a referente à estratégia HS.
Apesar disso, seu uso também requereu uma implementação, parcial, que fornecesse o pré-processamento da reserva de exemplos na forma de agrupamento hierárquico (Seção \ref{metohs}).

\section{Metas atingidas}\label{metas}
Um sumário dos objetivos desta tese de doutorado, delineados previamente na Seção \ref{intropropostas}, e seus respectivos resultados obtidos nos experimentos realizados são descritos a seguir:
\begin{itemize}
   \item \textbf{Identificação da existência de relações de adequação entre nichos de problemas e estratégias:} a presença dessas relações foi identificada por meio da análise qualitativa de árvores de decisão baseadas em meta-atributos humanamente interpretáveis; adicionalmente, foi constatado que a escolha do algoritmo de aprendizado precede as demais variáveis na determinação do sucesso da estratégia (Seção \ref{poralg}).
   \item \textbf{Desenvolvimento de uma estratégia capaz de suprimir, sob demanda, a influência do aprendiz:} a estratégia HTU foi proposta como uma alternativa para controlar a atuação do aprendiz na estratégia baseada em densidade, TU; ela se mostrou competitiva frente às demais estratégias consideradas e apresentou propriedades relevantes, como estabilidade, segurança e custo computacional interconsultas \textit{humanamente tolerável} (Seção \ref{cuscomp}) - características que permitem redução nos custos de esforço humano.
   \item \textbf{Desenvolvimento de um aprendiz meta-ativo:} a abordagem de recomendação automática proposta, chamada \textit{aprendizado meta-ativo}, \textit{superou}\footnote{A validade da avaliação do sucesso do meta-aprendizado é limitada pelo método experimental empregado. Algumas análises posteriores (Apêndice \ref{apflu}) apontam no sentido de ser necessário maior rigor experimental.}, no nível base, o uso de um único algoritmo de aprendizado específico e mostrou-se viável, no nível meta, com diversos algoritmos no papel de meta-aprendizes (principalmente PCT, RoF e RFw).
   Além da recomendação automática de algoritmos de aprendizado no contexto de aprendizado ativo, outras modalidades de recomendação também se mostraram promissoras (dentro das limitações metodológicas constatadas posteriormente), como a recomendação de estratégias e pares estratégia-algoritmo.
\end{itemize}
Tais resultados incluíram a comprovação das hipóteses, conforme descrito a seguir, na Seção \ref{conchipoteses}.

\section{Hipóteses comprovadas}\label{conchipoteses}

A hipótese principal, de que, em tarefas de classificação, relações entre conjuntos de dados, algoritmos de aprendizado ativo e estratégias de amostragem ativa podem ser exploradas visando um maior desempenho preditivo frente à escolha arbitrária é válida, dentro das limitações experimentais.
Essa conclusão se baseia na comparação de medidas relevantes de acurácia como: a ALC da medida $\mu_\kappa$; a correlação entre ranqueamentos preditos e esperados; e, a acurácia ordinária, a acurácia balanceada e a medida $\kappa$ no nível meta.

Da mesma forma, a hipótese secundária, de que o aprendiz pode ser automaticamente inibido ou substituído com vantagem durante o aprendizado, também foi demonstrada válida.
Essa conclusão se baseia na comparação da estratégia HTU com sua antecessora na literatura, cuja presença do aprendiz é constante, e também com estratégias representantes de outros paradigmas.
Também contribui para a validade da hipótese os resultados favoráveis da comparação do meta-aprendiz, ativado antes da 1\textordfeminine~e depois da 50\textordfeminine~consulta, com as outras estratégias no nível base e com valores de referência no nível meta.

As seguintes afirmações foram empiricamente verificadas, segundo o método experimental empregado e considerados os conjuntos de dados, o conjunto específico de estratégias e os algoritmos adotados.
\begin{enumerate}
\item HTUeuc tem o desempenho mais consistente, ou seja, é a mais segura em termos financeiros/de esforço humano.
\item ATUeuc e HTUeuc apresentam a menor variabilidade de desempenho preditivo ($\sigma_\kappa$).
\item Dentre as estratégias com melhor desempenho, ATUeuc e HTUeuc possuem o menor custo computacional entre consultas e, consequentemente, de esforço humano.
\item A recomendação automática de algoritmos aumenta o desempenho da estratégia.
% \item O meta-aprendiz induz modelos preditivos que representam o conceito subjacente à tarefa de recomendação automática de algoritmos de aprendizado para aprendizado ativo, com precisão suficiente para superar referências relevantes.
\item Diferentes algoritmos podem gerar metaclassificadores com desempenhos preditivos similares, seguindo a abordagem de recomendação automática proposta.
\item Embora requeira um método experimental mais rigoroso (Apêndice \ref{apflu}), a recomendação automática de algoritmos, estratégias ou pares estratégia-algoritmo inicialmente mostrou-se viável.
\end{enumerate}
Quanto às outras modalidades de recomendação automática, foi observado que pesquisas adicionais ainda são necessárias para que se possa chegar a conclusões melhor fundamentadas.

\section{Limitações}\label{limitacoes}

Durante e após a pesquisa realizada nesta tese, algumas limitações foram identificadas.
Elas não puderam ser superadas devido a diversos fatores, como baixa prioridade, percepção ou aparecimento tardio, impossibilidade prática, ausência de consenso, limitação intrínseca, entre outros.
% O desempenho do algoritmo na segunda metade da curva é afetado pela qualidade das consultas realizadas na primeira metade.
% Logo, seria ideal, para a segunda metade, construir um conjunto de metaexemplos em que as consultas sejam geradas com base em uma primeira metade construída pelo algoritmo recomendado ou pelo melhor.
% Porém, a forma adotada foi gerar as duas metades de consultas com o mesmo algoritmo.
% Requereria 2n execuções: para a primeira metade, como usual, mais n execuções para a segunda metade (n algoritmos).
% Da forma feita, foram n execuções: n pra curva toda mais 0.

Uma limitação comum de trabalhos na literatura de classificação também está presente neste.
Cada domínio de aplicação tem um modelo de custos de erro de classificação mais adequado e, consequentemente, requer uma medida de avaliação de desempenho apropriada.
Logo, a adoção uniforme da medida de desempenho no nível base ($\kappa$) para todos os conjuntos de dados configura-se como uma limitação metodológica.
Uma investigação aprofundada do domínio de cada conjunto de dados seria capaz de identificar a medida mais apropriada para cada conjunto, de acordo com as melhores práticas na literatura para o domínio do problema em questão.
A solução ideal seria que cada conjunto de dados pré-processado fosse disponibilizado com essa informação. Entretanto, em muitos casos não há um critério para isso.

A subamostragem mencionada na Seção \ref{despred} reduziu a reserva para 100 exemplos, visando maior tratabilidade computacional de EER (estratégia baseada na redução do erro esperado).
Tal decisão pode ter beneficiado essa estratégia.
Era esperada uma maior similaridade entre o comportamento de EER e o das outras estratégias não agnósticas.
Assim, idealmente, todas as estratégias deveriam compartilhar dessa manipulação na reserva para garantir uma comparação em iguais condições.
Embora essa parte do método experimental pudesse ser melhorado, seu provável impacto foi aumentar a competitividade de EER, limitando-se a reduzir a visibilidade do sucesso das propostas.

\citeonline{journals/pr/Lughofer12} publicou uma estratégia híbrida baseada em agrupamento da qual o autor deste documento não tomou conhecimento em tempo hábil para uma devida comparação com HTU.
Embora a comparação com aquela e as diversas outras estratégias não contempladas pelos experimentos pudesse enriquecer este trabalho, a sua ausência não impactou seriamente os resultados, pois o objetivo deste trabalho se concentrou mais nos problemas de escolha no cenário de aprendizado ativo, como a estratégia e o tipo ou o momento de inibição de algoritmos de aprendizado, do que no desempenho de estratégias específicas.

Uma limitação importante de HTU é seu parâmetro $\rho_{\limiar}$.
Apesar da superioridade de seu desempenho ter sido demonstrada experimentalmente,
a validade do princípio de funcionamento motivador de sua proposta não foi comprovada.
Seria necessário comparar HTU com uma estratégia híbrida aleatória ou conforme alguma heurística pré-definida, ou seja, com outros critérios de alternância entre ATU e TU que servissem de referência.
Um indício de que a medida de correlação de Pearson talvez não quantifique a grandeza desejada, como o nível de contribuição da componente exploratória, é que, na maioria dos casos testados, $\rho_{\limiar}<0$ levou a desempenhos melhores do que $\rho_{\limiar}>0$  - conforme apresentado previamente na Figura \ref{hist}.
Isso precisaria ser investigado em mais detalhes, pois contradiz a motivação da proposta.

O cenário adotado é artificial, caso sejam considerados alguns aspectos de aplicações reais.
Um desses aspectos é sobre os exemplos duplicados, ou seja, com os valores dos atributos coincidentes, mas com rótulos conflitantes.
Eles foram evitados nos experimentos para melhor isolamento do objeto de estudo (Capítulo \ref{metodologia}), apesar de serem um tópico ativo de pesquisa \cite{journals/datamine/IpeirotisPSW14,conf/kdd/ShengPI08}.

As estratégias DW, TU, ATU e HTU poderiam ter tido um melhor desempenho se os coeficientes de ponderação houvessem sido ajustados de acordo com alguma heurística, pois eles balanceiam a importância da densidade e do aprendiz.
Nos experimentos, esses parâmetros foram arbitrariamente mantidos com o valor 1.

\citeonline{conf/ijcnn/SoutoPSACLS08} reportaram resultados sobre recomendação de algoritmos de agrupamento possivelmente mais efetivos, embora trate-se de outro domínio e outro aparato experimental.
Segundo esses autores, as predições do meta-aprendiz atingiram um valor médio de correlação 0,75 contra o valor 0,59, de referência.
O conjunto de meta-atributos proposto por eles poderia ter sido adotado.
% Method SRC
% Default 0.59 +- 0.37
% Meta-Leaner 0.75 +- 0.21

Um dos pressupostos do sistema de recomendação automática proposto é que o conjunto inicial de treinamento seria pequeno demais para que fosse possível realizar uma seleção de modelos que visasse a escolha do melhor algoritmo de aprendizado.
Realmente, a quantidade de apenas $|Y|$ exemplos antes do início do processo de rotulação inviabiliza qualquer tentativa de seleção via validação cruzada.
Entretanto, após 50 consultas, a possibilidade de seleção do melhor algoritmo por meio de validação cruzada poderia ter sido verificada e comparada com a seleção automática.
Ainda que 50 exemplos venham a ser insuficientes, espera-se que, em algum ponto da curva de aprendizado, dado o crescimento do conjunto de treinamento, a convencional seleção por meio de validação cruzada se torne preferível à seleção automática.
Trata-se, assim, de uma referência a ser adotada em futuras comparações. 
Quanto maior o orçamento, mais importante se torna essa referência.

Por fim,
uma limitação metodológica, presente também em outros trabalhos, diz respeito à escolha ideal da coleção de conjuntos de dados.
Mesmo com o procedimento de eliminação de conjuntos muito similares, a dependência remanescente entre conjuntos teve uma influência quantificável nos resultados.
Essa influência é analisada no Apêndice \ref{apflu}.
Resumidamente, as medidas de desempenho foram elevadas, em parte, por conta dessa parcial \textit{dependência entre amostras} (conjuntos de dados similares).
Apesar disso, a recomendação automática na parcela mais independente da coleção também superou a referência, mas em muito menor grau.
Embora a presença de domínios similares tenha facilitado a tarefa de recomendação, isso não reduz a necessidade prática do sistema de recomendação - apenas sugere que ele seja, naturalmente, mais efetivo em coleções que contenham conjuntos cujos domínios sejam próximos ao do conjunto que se pretenda rotular.
Diante dessa redução na generalidade dos resultados obtidos, \destaque{ainda não é possível afirmar com segurança que meta-aprendizado seja a melhor forma de resolver os problemas de escolha envolvidos no aprendizado ativo.} 

\section{Desdobramentos Futuros}\label{futuro}
Um aprofundamento nas análises dos experimentos ainda é necessário.
Trata-se de uma abordagem cujo princípio de funcionamento e efetividade requerem análises ainda mais rigorosas do que aquelas empregadas neste documento.
Por exemplo, na Figura \ref{dsscorr} do Apêndice \ref{apflu}, é possível identificar os conjuntos de dados mais difíceis para a tarefa de recomendação de algoritmos de aprendizado.
Os motivos de cada dificuldade podem abrir possibilidades de melhoria nas propostas de recomendação automática, caso sejam identificados; ou, podem indicar não tratar-se de um assunto prioritário de pesquisa.
Uma análise inicial indica que foram favorecidos justamente os conjuntos que dispuseram da presença de outros conjuntos do mesmo domínio no metaconjunto de treinamento.
Logo, a coleção, mesmo sendo diversa, na realidade não continha conjuntos cuja extração de meta-atributos pudesse ser útil no aproveitamento de conhecimento entre domínios.

% o desempenho geral dos algoritmos se torna menos distinguível à medida que o aprendizado avança, conforme visto previamente na Figura \ref{curvasrankbands} no intervalo de consultas considerado. 

O autor desta tese considera a exploração das potenciais abordagens situadas na intersecção entre as áreas de aprendizado ativo e meta-aprendizado ainda em seus primórdios, pois abrange uma lacuna literária a ser preenchida.
De fato, há margem para melhorias significativas conforme sugere a Figura \ref{curvasrankbandsmetabest}.
\begin{figure}[]
\centering
	\includestandalone[mode=buildmissing]{bandsmetabest}
	\caption[Curvas de ranqueamento com meta-aprendiz e recomendação perfeita.]{Curvas de ranqueamento - incluindo recomendação perfeita. Medida comparada: $\mu_{\kappa}$. \textit{Detalhes na Figura \ref{curvasrankbands}.}}
	\label{curvasrankbandsmetabest}
\end{figure}
Nessa figura, é possível observar a faixa de colocações obtidas pelas estratégias quando dispõem do meta-aprendiz perfeito (metaBest), isto é, aquele que é capaz de predizer, com acesso desleal às metaclasses, sempre o melhor algoritmo nas duas metades do intervalo de consultas.
Até mesmo o limite inferior da faixa, que normalmente corresponde à pior estratégia, foi capaz de superar as demais abordagens por praticamente todo o período.
Embora a faixa de metaBest certamente seja fruto de sobreajuste aos dados, ela permite verificar que a curva de metaPCT não é pressionada pelo limite do que é teoricamente possível.

A opção por dois momentos de recomendação automática, um para cada metade do período de consultas, foi arbitrária.
A existência de um vale precisamente em torno do momento de transição (metade do orçamento, $\cent=50$)
sugere que o momento ideal para a reconsideração do melhor algoritmo varie de um conjunto de dados para outro - esse comportamento já havia se manifestado nos experimentos da Seção \ref{desmetap}.
Por exemplo, na vigésima consulta, a curva de metaBest começa a ceder colocações; conforme a quantidade de consultas aumenta, maior a quantidade de conjuntos de dados em que algoritmos distintos do inicialmente escolhido tornam-se os mais adequados.
Dado que a troca de algoritmo foi definida para ocorrer apenas naquele ponto (50\textordfeminine\xspace consulta), é somente a partir dali que as estratégias puderam usufruir de um algoritmo mais adequado para o intervalo de consultas corrente.
Essa constatação poderia ser explorada adicionando-se mais momentos em que o algoritmo de recomendação pudesse reconsiderar a escolha do algoritmo do aprendiz.
No caso limite, o algoritmo do aprendiz poderia ser reconsiderado dinamicamente a cada consulta, tal como a abordagem de \citeonline{rossi2014meta} em fluxos de dados; porém, com um custo computacional $\cent$ vezes maior na etapa de treinamento do algoritmo de recomendação.
Dessa forma, uma extensão natural do trabalho seria a recomendação automática constante, isto é, a cada nova consulta.
O limite teórico aumentaria consideravelmente para as curvas que fossem obtidas com essa eventual extensão do trabalho.

Um conjunto reduzido com apenas os meta-atributos mais relevantes pode ser investigado, de forma a aumentar a acurácia preditiva.
Seria interessante também, conforme sugestão da Banca, investigar qual o padrão dos conjuntos de dados que fazem com que seja preferível a amostragem aleatória.

Outro desdobramento, mais imediato, seria a implementação da \textit{metaestratégia} e, possivelmente, do \textit{metapar estratégia-algoritmo}, para que a viabilidade dessas modalidades pudesse ser confirmada também no nível base, por meio de comparações das ALCs e das curvas de ranqueamento.

Por fim, outra possibilidade de meta-aprendizado aplicado a aprendizado ativo é a recomendação de parâmetros de estratégias.



















\input tex/nota