Apesar do crescente avanço no desenvolvimento de algoritmos capazes de induzir modelos preditivos
numa grande diversidade de domínios, entraves de cunho econômico, dentre outros, ainda persistem.
Embora tais modelos sejam construídos sem programação explícita,
eles não podem, em geral, dispensar a supervisão humana no processo inicial de descoberta de rótulos que
normalmente se atribuem a certas parcelas dos dados disponíveis.
Em vista da forte possibilidade dos dados serem massivos,
as parcelas não podem ser proporcionais ao total de dados.
Elas devem permanecer pequenas, dentro do devido orçamento.
Isso propõe um compromisso entre o custo de rotulação e a acurácia preditiva,
cuja solução pode ser a adoção de uma técnica de aprendizado ativo.

Existem diversas abordagens de se amostrarem ativamente os dados para rotulação.
Diferentemente do aprendizado passivo, não é possível testá-las para posterior escolha da melhor
segundo alguma métrica de acurácia.
A cada rótulo obtido incorre-se em custo adicional.
Dessa forma, a situação ideal seria a ciência prévia da melhor estratégia
para o domínio em questão, mesmo com poucos rótulos conhecidos de antemão;
ou, em maior detalhe,
para cada instante durante a aquisição de rótulos naquele domínio.

Nesta tese, a viabilidade do aprendizado ativo é comprovada empiricamente com
solidez estatística 
de acordo com três pontos de vista: adaptação e comparação dos principais
paradigmas e de sua efetividade em geral;
na definição de nichos adequados para cada estratégia;
e, na demonstração de que é possível definir previamente as estratégias mais
adequadas automaticamente. \ano{conferir}
É também empreendida uma análise amplamente negligenciada pela
literatura da área:
o risco devido à variabilidade dos algoritmos.

% o viés de amostragem tem seus perigos, então quando não há diferença estatística entre aleatório e AA, deve-se prefirir aleatório,
% ou melhor, cluster-based, por suas garantias estatisticas peculiares.

\newpage
\thispagestyle{empty}
\mbox{}