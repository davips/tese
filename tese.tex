\documentclass[12pt,twoside]{book}
% \tracingmacros2
\usepackage{etex}
\usepackage{afterpage}
\usepackage[utf8]{inputenc}
% \inputencoding{latin1}

\usepackage{scrextend}
% \usepackage{cite}
% \usepackage[pdftex]{graphicx}
% \usepackage{array}
\usepackage{amsmath}
\usepackage{wasysym}
% \usepackage{mathtools}

\usepackage{bm}
% \usepackage{stmaryrd}
% \usepackage[tight,footnotesize]{subfigure}

\DeclareMathOperator*{\argmax}{arg\,max}
\DeclareMathOperator*{\argmin}{arg\,min}
\DeclareMathOperator*{\sk}{sk}
\DeclareMathOperator*{\ku}{ku}
\DeclareMathOperator*{\val}{val}
\DeclareMathOperator*{\nom}{nom}
\DeclareMathOperator*{\av}{av}
\DeclareMathOperator*{\sd}{sd}
\DeclareMathOperator*{\en}{en}
\DeclareMathOperator*{\co}{co}
\DeclareMathOperator*{\dime}{dim}

\newcommand{\meutitulo}{Aprendizado ativo ... }
\newcommand{\meunome}{Davi Pereira dos Santos}
\newcommand{\meuorientador}{Prof. Dr. André Carlos Ponce de Leon Ferreira de Carvalho}
\newcommand{\minhadata}{Dezembro/2014}
\newcommand{\minhabolsa}{Trabalho realizado com o apoio da CAPES.}
\newcommand{\meugrau}{Doutor}

%portugues
\usepackage[ruled,linesnumbered,vlined]{algorithm2e}
\usepackage{fancyhdr}

\usepackage{ textcomp }

%\renewcommand{\listalgorithmcfname}{Lista de Algoritmos}%
%\renewcommand{\algorithmcfname}{Algoritmo}%
\usepackage[english,brazil]{babel}
\usepackage{subfigure}
%%\usepackage{epigraph}
%%\usepackage{endnotes}
%\renewcommand{\notesname}{Comentários}
\usepackage{natbib}
\usepackage{latexsym}
\usepackage{setspace}
\usepackage{xspace}
% \usepackage[nohyperlinks]
%%% referencias com página \vref
\usepackage[brazil]{varioref}
\usepackage{indentfirst}
%%% figuras um ao lado do outro
\usepackage{subfigure}
%%% definir títulos de seção
\usepackage[sf,sl,outermarks]{titlesec}
\titleformat{\section}{\color{black}\normalfont\Large\bfseries}{\color{darkgray}\thesection}{1em}{}
\titleformat{\subsection}{\color{black}\normalfont\bfseries}{\color{darkgray}\thesubsection}{1em}{}

%%% boxes
\usepackage{fancybox}
\usepackage{fancyvrb}
\usepackage[usenames,dvipsnames]{color}
%\usepackage[pdftex]{graphicx}
\usepackage{graphicx}
\usepackage[pdftex]{geometry}
  \geometry{a4paper,left=3cm,right=2cm,top=2.0cm,bottom=2cm,twoside}

%%% cria links no arquivo .pdf
%%% comentar as linhas na versão para impressao
\usepackage[pdftex,pdfpagelabels,pagebackref,pageanchor=false]{hyperref}

% % Para imprimir PDFs com offset
% \begin{document}
% \begin{figure} %[H]
%     \centering
%     \includegraphics[bb=640 0 0 850]{/home/davi/12.pdf}
% %     \caption{\textit{Generative ensembles}}
% %     \label{gen}
% \end{figure}
% \end{document}


\usepackage{pdfpages}
\usepackage{multirow}
\usepackage{makecell}
\hypersetup{
    pdftitle = {\meutitulo},
    pdfsubject = {},
    pdfkeywords = {},
    pdfauthor = {\meunome}
    }
\hypersetup{colorlinks=true,linkcolor=blue,citecolor=blue,hypertexnames=false}
% \usepackage[adobe-utopia]{mathdesign} %pacote texlive-fonts-extra no ubuntu
\usepackage{ae}
% \usepackage{bookman}
\usepackage[T1]{fontenc}
\usepackage{amsfonts}
\DeclareFixedFont{\numberfont}{T1}{phv}{bx}{n}{2cm}
\titleformat{\chapter}[display]
  {\normalfont\Large\sffamily
  }
  {%\titlerule[3pt]%
   \filright
   \rule[32pt]{.7\linewidth}{4pt}
   \hspace{-11pt}
   \shadowbox{
   \begin{minipage}{.18\linewidth}
     \begin{center}
       \textsc{\Large\chaptertitlename}\\
       \vspace{1ex}
       {\numberfont\color[gray]{0.5} \thechapter}\\
       \vspace{1ex}
     \end{center}
   \end{minipage}}
  }
  {0pt}
  {\filcenter
   \Huge
   }
  [\hfill\rule{.8\textwidth}{0.5pt}\\
     \vskip-1.8ex\hfill\rule{.7\textwidth}{3pt}]

\newcommand{\versal}[1]{{\noindent
    \setbox0\hbox{\largefont #1 }%
    \count0=\ht0                   % height of versal
    \count1=\baselineskip          % baselineskip
    \divide\count0 by \count1      % versal height/baselineskip
    \dimen1 = \count0\baselineskip % distance to drop versal
    \advance\count0 by 1\relax     % no of indented lines
    \dimen0=\wd0                   % width of versal
    \global\hangindent\dimen0      % set indentation distance
    \global\hangafter-\count0      % set no of indented lines
    \hskip-\dimen0\setbox0\hbox to\dimen0{\raise-\dimen1\box0\hss}%
    \dp0=0in\ht0=0in\box0}}

\DeclareMathAlphabet{\mathcal}{OMS}{cmsy}{m}{n}
\usepackage{pbox}
\usepackage{marvosym}
\usepackage{bm}
\usepackage{wasysym}
\usepackage{makecell}
\usepackage{graphicx}
\usepackage{icomma}
\usepackage{afterpage}
\usepackage{pdflscape}
\usepackage{rotating}
% \usepackage{subcaption}
\usepackage{pgfplots}
\usepackage{amsopn}
\usepackage{soul}

\pgfplotsset{compat=1.10}
\usetikzlibrary{trees}
\usetikzlibrary{patterns}
\usepgfplotslibrary{fillbetween}
\usetikzlibrary{fadings}
\tikzfading[name=myfading, top color=transparent!0, bottom color=transparent!80]
%melhor estilo pra 1a pg do capítulo
\chapterstyle{ell}
% sumario melhor
% \renewcommand{\cftchapterfont}{\sffamily}   
\renewcommand{\cftsectionfont}{\normalfont\sffamily}  
\renewcommand{\cftsubsectionfont}{\itshape\sffamily} 
\settocdepth{section}
% headline melhor pras seções e capítulo
\renewcommand*{\chapnumfont}{\bfseries\HUGE\sffamily}
\renewcommand*{\chaptitlefont}{\normalfont\huge\sffamily}
\setsecheadstyle{\bfseries\Large}
\setsubsecheadstyle{\bfseries\normalsize}

\DeclareMathOperator*{\unidist}{U}
\DeclareMathOperator*{\cov}{cov}
\DeclareMathOperator*{\argmax}{arg\,max}
\DeclareMathOperator*{\argmin}{arg\,min}
\DeclareMathOperator*{\sk}{sk}
\DeclareMathOperator*{\ku}{ku}
\DeclareMathOperator*{\qnc}{\#nc}
\DeclareMathOperator*{\qnom}{\#no}
\DeclareMathOperator*{\qnum}{\#nu}
\DeclareMathOperator*{\qatt}{\#at}
\DeclareMathOperator*{\qexe}{\#ex}
\DeclareMathOperator*{\qexeatt}{\#ea}
\DeclareMathOperator*{\lgqexe}{lgex}
\DeclareMathOperator*{\lgqexeatt}{lgea}
\DeclareMathOperator*{\nom}{isnom}
\DeclareMathOperator*{\pno}{\%no}
\DeclareMathOperator*{\en}{en}
\DeclareMathOperator*{\corr}{cr}
\DeclareMathOperator*{\cn}{cn}
\DeclareMathOperator*{\cnk}{cnk}
\DeclareMathOperator*{\si}{si}
\DeclareMathOperator*{\sik}{sik}
\DeclareMathOperator*{\du}{du}
\DeclareMathOperator*{\duk}{duk}
\DeclareMathOperator*{\dime}{dim}
\DeclareMathOperator*{\info}{Inf_\theta}
\DeclareMathOperator*{\stratID}{ID_\theta}
\DeclareMathOperator*{\stratIDTU}{ID_{TU_\theta}}
\DeclareMathOperator*{\JS}{JS}
\DeclareMathOperator*{\at}{\bm{a}}
\DeclareMathOperator*{\simi}{sim}
\DeclareMathOperator*{\limiar}{limite}
\DeclareMathOperator*{\valor}{-valor}
\DeclareMathOperator*{\limiaring}{max}

\definecolor{dark-green}{rgb}{0,0.5,0}
\newcommand{\parei}[1]{\phantom{vvvvvvvvvvvvvvvvvvvvvvvvvvvvvvvvvvvvvvvvvvvvvvvvvvvvvvvvvvvvvvv}
\ano{--------------- parei aqui: #1 ------------------}
\phantom{vvvvvvvvvvvvvvvvvvvvvvvvvvvvvvvvvvvvvvvvvvvvvvvvvvvvvvvvvvvvvvv}}
\newcommand{\destaque}[1]{\textbf{\textit{#1}}}
\newcommand{\opar}{min5NNw}
\newcommand{\X}{X\xspace}
\newcommand{\Y}{Y\xspace}
\newcommand{\xt}{\bm{x}_{(t)}\xspace}
\newcommand{\x}{\bm{x}\xspace}
\newcommand{\yt}{y_{(t)}\xspace}
\newcommand{\pool}{reserva de exemplos\xspace}
\newcommand{\pools}{reservas de exemplos\xspace}
\newcommand{\Upool}{\mathcal{U}\xspace}
\newcommand{\Ut}{\U_{(t)}\xspace}
\newcommand{\Lt}{\mathcal{L}_{(t)}\xspace}
\newcommand{\tra}[2]{\underline{#1}\xspace\footnote{\textit{#2}\xspace}}
\newcommand{\blue}[1]{\textcolor{blue}{#1}\xspace}
\newcommand{\red}[1]{\textcolor{red}{#1}\xspace}
\newcommand{\green}[1]{\textcolor{dark-green}{#1}\xspace}
\newcommand{\esb}[1]{\blue{#1}\xspace}
\newcommand{\ano}[1]{\red{[$\star$ #1 $\star$]}\xspace}
\newcommand{\tar}[1]{\green{$\star$ #1}\xspace}
\newcommand{\versionspace}{espaço de versões\xspace}
\newcommand{\come}[1]{\green{\footnotesize\phantom{i}$\triangleleft$\phantom{i}\textit{#1}\xspace}}

\newcommand{\ing}[2]{\emph{#1}\footnote{[\textit{#2}]}}
\newcommand{\novo}[1]{\emph{#1}}
% \newcommand{\eer}{redução do erro esperado\xspace}
% \newcommand{\Eer}{Redução do erro esperado\xspace}
\newcommand{\elms}{máquinas de aprendizado extremo\xspace}
\newcommand{\elm}{máquina de aprendizado extremo\xspace}
\newcommand{\svm}{máquina de vetores de suporte\xspace}
\newcommand{\bom}[1]{\textcolor{blue}{\textbf{#1}}\xspace}
\newcommand{\bomd}[1]{\textbf{#1}\xspace}
\newcommand{\ruim}[1]{\textcolor{red}{\textbf{#1}}\xspace}

\newcommand{\tarefa}[1] {\addcontentsline{toc}{section}{\tar{#1}}}

\hyphenation{su-per-vi-sio-na-dos}
\hyphenation{su-per-vi-sio-na-do}
\DeclareMathOperator*{\stratIDATU}{ID_{ATU}}



\pgfplotsset{marst/.style={solid, violet, mark=*, mark options={solid, scale=1.4}, mark repeat=60}}
\pgfplotsset{sgst/.style={solid, thick, orange, mark=x, mark options={solid, scale=2}, mark repeat=60}}
\pgfplotsset{eerst/.style={solid, red, mark=square*, mark options={solid, scale=1.4}, mark repeat=60}}
\pgfplotsset{hsst/.style={dashed, black, mark=triangle*, mark options={solid, scale=1.4}, mark repeat=60}}
\pgfplotsset{tust/.style={solid, gray, mark=square, mark options={solid, scale=1.4}, mark repeat=60]}}
\pgfplotsset{rndst/.style={loosely dashed, olive, mark=triangle, mark options={solid, scale=1.4}, mark repeat=60}}
\pgfplotsset{atust/.style={dotted, mark=star, ultra thick, blue,  mark repeat=50, mark options={solid, scale=1.9}, mark repeat=60}}
\pgfplotsset{htust/.style={solid, thick, green!50!black, mark=10-pointed star, mark options={solid, scale=1.9}, mark repeat=60}}
\pgfplotsset{dwst/.style={solid, thick, gray, mark=diamond, mark options={solid, scale=1.4}, mark repeat=60}}
\pgfplotsset{eer2st/.style={solid, brown, mark=pentagon, mark options={solid, thick, scale=1.4}, mark repeat=60}}
\pgfplotsset{htu2st/.style={dotted, blue, mark=triangle, mark options={solid, scale=1.5}, mark repeat=60}}
\pgfplotsset{defst/.style={loosely dashed, thick, blue, no marks}}

\usepackage{alltt}
\usepackage{xcolor}
\usepackage{lmodern}
% \usepackage{parallel} % para as equacoes de medidas multilabel

%%%%%%%%%%%%%%%% FORMATACAO DE PALAVRAS %%%%%%%%%%%%%%%%%%%%%

\newcommand{\nf }[1] % nome de ferramenta
    {\textsc{#1}}
\newcommand{\pe }[1] % termo em ingles
    {\textit{#1}}

%%%%%%%%%%%%%%%%%%%%%%%%%%%%%%%%%%%%%%%%%%%%%%%%%%%%%%%%%
\usepackage{ctable}
\newcommand{\RAKEL}{{\emph {RAKEL}}\/\xspace}
\newcommand{\win}{{+}\/\xspace}
\newcommand{\lose}{{-}\/\xspace}

% \usepackage[usenames,dvipsnames]{xcolor}
\definecolor{cinzaclaro}{rgb}{0.84,0.84,0.84}
\newcommand{\e}{\colorbox{cinzaclaro}}

\usepackage{amssymb}
\usepackage[latin1]{inputenc}
\usepackage{pgfplots,bm}
\usetikzlibrary{plotmarks}
\usetikzlibrary{scopes,shapes,shadows,calc,arrows,shapes.symbols,automata,decorations.pathmorphing,backgrounds,fit,positioning}

% styles for flowcharts
\tikzstyle{d} = [diamond, draw, text width=4.5em, text badly centered, node distance=3cm, inner sep=0pt]
\tikzstyle{block} = [rectangle, draw, text width=3em, text centered, rounded corners, minimum height=4em]
\tikzstyle{s} = [-triangle 60,thick]
\tikzstyle{b} = [rectangle, thick, draw, aspect=3, shape border rotate=90, minimum height=1, rounded corners, minimum width=1, outer sep=-0.5\pgflinewidth, color=black, top color=black!10, bottom color=black!10, middle color=white]
\tikzstyle{ensemble} = [text height=1, rectangle, thick, draw, aspect=3, shape border rotate=90, minimum height=100, rounded corners, minimum width=100, outer sep=-0.5\pgflinewidth, color=black, top color=black!10, bottom color=black!10, middle color=white]
\tikzstyle{aprendizst} = [text height=1, rectangle, thick, draw, aspect=3, shape border rotate=90, minimum height=10, rounded corners, minimum width=10, outer sep=-0.5\pgflinewidth, color=black, top color=black!15, bottom color=white]
\tikzstyle{bdtr} = [cylinder, thick, draw, aspect=1, shape border rotate=90, minimum height=45, minimum width=40, outer sep=-0.5\pgflinewidth, color=black, left color=blue!40, right color=blue!40, middle  color=white]
\tikzstyle{bdts} = [cylinder, thick, draw, aspect=1, shape border rotate=90, minimum height=45, minimum width=40, outer sep=-0.5\pgflinewidth, color=black, left color=orange!40, right color=orange!40, middle  color=white]

\def\ctr [#1,#2]#3{
  \node (#1) [bdtr, #3] #2 {$\mathcal{L}$} ;
}

\def\cts [#1,#2]#3{
  \node (#1) [bdts, #3] #2 {$\mathcal{U}$} ;
}

\tikzstyle{nuvem}=[cloud, thick, draw, cloud puffs=12, aspect=5, alias=cyl, shape border rotate=90, minimum height=50, minimum width=60, outer sep=-0.5\pgflinewidth, color=black, top color=orange!40, bottom color=orange!40, middle  color=white]
\def\fluxo [#1,#2]#3{
  \node (#1) [nuvem, #3] #2 {fluxo de dados};
}
\pgfdeclarelayer{ss3} \pgfdeclarelayer{ss2} \pgfdeclarelayer{ss1}
\pgfdeclarelayer{pi3} \pgfdeclarelayer{pi2} \pgfdeclarelayer{pi1}
\pgfdeclarelayer{main}
\pgfsetlayers{ss3,ss2,ss1,main,pi1,pi2,pi3}

\tikzset{seta/.style={solid, ultra thick, black}}
\tikzset{setatr/.style={solid, ultra thick, blue}}
\tikzset{setats/.style={dashed, ultra thick, orange}}

\newcommand*\C[1]{
  \begin{tikzpicture}
      \node[draw,circle,inner sep=1pt] {#1};
  \end{tikzpicture}
}
\newcommand*\Q[1]{
  \begin{tikzpicture}
      \node[draw,rectangle,inner sep=2pt] {#1};
  \end{tikzpicture}
}
\newcommand*\T[1]{
  \begin{tikzpicture}
%       \node[draw,circle,inner sep=2pt] {#1};
      \node[draw,circle,inner sep=1.5pt] {#1};
      \node[draw,circle,inner sep=2.5pt] {#1};
       \node[black, thick, inner sep=2pt] {#1};
  \end{tikzpicture}
}

\def\estrutura [#1,#2]#3{
	\node (#1) [#3] #2{
	\begin{tikzpicture}[node distance=90, auto, >=stealth, scale=0.15, transform shape]
	  \node (ei#1) [d, thick] {};
	  \node (ei2#1) [below right of=ei#1, block, thick] {};
	  \node (ei3#1) [below left of=ei#1, block, thick] {};
	  \draw[->, seta] (ei#1.south east)  -- node[left=1] {} (ei2#1.west);
	  \draw[->, seta] (ei#1.south west)  -- node[left=1] {} (ei3#1.east);
	  \begin{pgfonlayer}{ss1}
	    \node (ax#1)  [draw, circle, fit=(ei#1) (ei2#1) (ei3#1), inner sep=1, scale=1, thick, top color=white, bottom color=white, middle color=black!10] {};
	  \end{pgfonlayer}
	\end{tikzpicture}
	};
}

\def\fronteira [#1,#2]#3{
	\node (#1) [#3] #2{
	\begin{tikzpicture}[node distance=0, auto, >=stealth, scale=0.2, transform shape]
		      \pgfplotsset{width=7cm}
		      \pgfplotsset{samples at={30.1,30.2,...,65}}
					\begin{axis} [name=ei#1,xmin=30, xmax=70, ymin=1.5, ymax=1.85, axis y line=left, axis x line=bottom] %,xlabel=peso (kg),   ylabel=altura (m)]
					\addplot[only marks,mark=*,mark options={blue, ultra thick, scale=5}] plot coordinates {				  (65,1.8) (54,1.75) (63,1.6) (72,1.65) (50,1.8) (62,1.68)  (77,1.71)				      };
					\addplot[only marks,mark=o,mark options={ultra thick,red,scale=5}] plot coordinates {				  (35,1.7) (39,1.60) (50,1.60) (58,1.54) (46,1.67) (43,1.55)				      };
					\addplot [only marks, mark=*, mark options={teal, ultra thick,scale=2}] {2.36 -x/(131/1.7)};
					\end{axis}
		\begin{pgfonlayer}{ss1}
			  \node (ax#1)  [draw, circle, fit=(ei#1), inner sep=1, scale=1, thick, top color=white, bottom color=white, middle color=black!10] {};
		\end{pgfonlayer}
	\end{tikzpicture}
	};
}


\def\algtrein[#1,#2,#3]#4{
		\node (#1) [b, label=-90:
% \begin{tabular}{c}
  alg. trein.#3
% \end{tabular}
, #4] #2{
			$a_{#3}( \theta_#3^{(t-1)}, \langle \bm{x},y \rangle )$
		};
}

\def\classificador[#1,#2,#3]#4{
		\node (#1) [b, label=-90:$h_#3^{(t)}$, #4] #2{
				$\argmax_y\{ P_{\theta_#3} (y|\bm{x})\}$
		};
}

\def\classificadorprob[#1,#2,#3]#4{
		\node (#1) [b, label=-90:${ }_#3$, #4] #2{
				$\max_{y \in Y}\{P_{\theta_#3} (y|\bm{x})\}$
		};
}

\def\aprendiz [#1,#2,#3]#4{
	\node (#1) [label distance=-2, #4,label=-90:aprendiz passivo #3] #2{
	\begin{tikzpicture}[node distance=0, auto, scale=1, transform shape,remember picture, text centered]
				\algtrein [a#1,  , #3] {}
				\tikzset{label distance=-8, node distance=85}
				\estrutura [f#1, ] {label=-90:$\theta_{#3}^{(t)}$, right of=a#1}
				\tikzset{label distance=-2, node distance=100}
				\classificador [c#1, , #3] {right of=f#1}
				\draw[->,setatr] (a#1) -- node (1,1)[] {} (f#1);
 			   \draw[<->, setats] (f#1)  -- node[left=1] {} (c#1);
%  			   \draw[->, ultra thick] (c#1)  to [out=110,in=60] node[] {} (f#1);
			\begin{pgfonlayer}{ss2}
			\node (aa#1) [aprendizst, fit = (a#1) (f#1)(c#1), inner sep=10] {} ;
			\end{pgfonlayer}
%  		   \draw[->, ultra thick] (aa#1.west)  -- node[] {} (a#1);
	\end{tikzpicture}
	};
}

\def\homem [#1,#2]#3{
  \node (#1) [#3] #2{
  \begin{tikzpicture}[node distance=0, auto, >=stealth, scale=0.77, transform shape]
    \node (cabeca) [draw, circle, inner sep=3, ultra thick] {. .};
    \node (umbigo) [circle,below=1.2 of cabeca] {};
    \node (tronco) [circle,below=0.3 of cabeca] {};
    \draw[-, seta] (cabeca.south)  to [out=-90,in=90] node[] {} (umbigo);
    \node (pesq) [circle,below left=0.5 of umbigo] {};
    \node (pdir) [circle,below right=0.5 of umbigo] {};
    \draw[-, seta] (pesq)  to  node[] {} (umbigo.north);
    \draw[-, seta] (pdir)  to  node[] {} (umbigo.north);
    \node (mesq) [circle,below left=0.5 of cabeca] {};
    \node (mdir) [circle,below right=0.5 of cabeca] {};
    \draw[-, seta] (mesq)  to  node[] {} (tronco.north);
    \draw[-, seta] (mdir)  to  node[] {} (tronco.north);
    \begin{pgfonlayer}{ss2}
      \node (fit#1) [b, draw, fit = (cabeca) (pesq) (pdir)] {};
    \end{pgfonlayer}
  \end{tikzpicture}
  };
}

\def\modeloprob [#1,#2,#3]#4{
	\node (#1) [label distance=-2, #4,label=-90:aprendiz #3] #2{
	\begin{tikzpicture}[node distance=0, auto, scale=1, transform shape,remember picture, text centered]
				\algtrein [a#1,  , #3] {}
				\tikzset{label distance=-8, node distance=85}
				\estrutura [f#1, ] {label=-90:$\theta_#3^{(t)}$, right of=a#1}
				\tikzset{label distance=-2, node distance=100}
				\classificadorprob [c#1, , #3] {right of=f#1}
				\draw[->,setatr] (a#1) -- node (1,1)[] {} (f#1);
 			   \draw[<->, setats] (f#1)  -- node[left=1] {} (c#1);
%  			   \draw[->, ultra thick] (c#1)  to [out=110,in=60] node[] {} (f#1);
			\begin{pgfonlayer}{ss2}
					\node (aa#1) [aprendizst, fit = (a#1) (f#1)(c#1), inner sep=12] {} ;
			\end{pgfonlayer}
%  		   \draw[->, ultra thick] (aa#1.west)  -- node[] {} (a#1);
	\end{tikzpicture}
	};
}

% version space
\def\modelovs [#1,#2,#3]#4{
	\node (#1) [label distance=-2, #4,label=-90:aprendiz #3] #2{
	\begin{tikzpicture}[node distance=0, auto, scale=1, transform shape,remember picture, text centered]
				\algtrein [a#1,  , #3] {}
				\tikzset{label distance=-8, node distance=85}
				\estrutura [f#1, ] {label=-90:$\theta_#3^{(t)}$, right of=a#1}
				\tikzset{label distance=-2, node distance=100}
				\node (c#1) [b, right of=f#1, label=-90:$\textit{version space}$] {
  \begin{tabular}{ll}
    $\bf{x}$ que melhor define \\
    hip�teses v�lidas
  \end{tabular}
};
				\draw[->,setatr] (a#1) -- node (1,1)[] {} (f#1);
%  			   \draw[->, ultra thick] (c#1)  to [out=110,in=60] node[] {} (f#1);
			\begin{pgfonlayer}{ss2}
					\node (aa#1) [aprendizst, fit = (a#1) (f#1)(c#1), inner sep=12] {} ;
			\end{pgfonlayer}
%  		   \draw[->, ultra thick] (aa#1.west)  -- node[] {} (a#1);
	\end{tikzpicture}
	};
}

% \usepackage{glosstex}
\usepackage[printonlyused,withpage]{acronym}

\usepackage{doi}
\usepackage{pbox}
\usepackage{amsfonts}
\usepackage{icomma}

\usepackage{pdflscape}
\usepackage{capt-of}
\begin{document}
\definecolor{darkgreen}{rgb}{0.0, 0.4, 0.0}
\pagestyle{empty}
\pagenumbering{roman}
\newcommand{\titulo}{\meutitulo}
\newcommand{\autor}{\meunome}
\newcommand{\orientador}{\meuorientador}
\newcommand{\nota}{Tese apresentada ao Instituto de Ciências Matemáticas e de Computação - ICMC-USP, para o Exame de Qualificação, como parte dos requisitos necessários à obtenção do título de \meugrau\xspace em Ciências de Computação e Matemática Computacional.}
\newcommand{\data}{\minhadata}
\newcommand{\comentario}{\minhabolsa}

 \begin{titlepage}


  \begin{tikzpicture}[remember picture,overlay]
    \node[inner sep=0pt,scale=1.06] at (current page.center) {
      \includegraphics[page=1]{SECAO-POSGRAD_87_Modelo_Capa_DO_CCMC_ORIGINAL1.pdf}
    };
  \end{tikzpicture}

  \cleardoublepage
  \newpage
  \clearpage

  \begin{tikzpicture}[remember picture,overlay]
    \node[inner sep=0pt,scale=1.06] at (current page.center) {
      \includegraphics[page=1]{SECAO-POSGRAD_87_Modelo_Capa_DO_CCMC_ORIGINAL2.pdf}
     };
  \end{tikzpicture}

% \includepdf[pages=-,noautoscale]{imagens/capa1.pdf}
% \includepdf[pages=-,noautoscale]{imagens/capa2.pdf}

% \end{document}\begin{figure} %[H]
%     \centering
%     \includegraphics[bb=640 0 0 740]{imagens/capa2.pdf}
% \end{figure}
%
% \cleardoublepage
%
% \begin{figure} %[H]
%     \centering
%     \includegraphics[bb=640 100 0 850]{imagens/capa1.pdf}
% \end{figure}


% \ \vfill
%
% \begin{center}
% \begin{minipage}[c]{12cm}
% \begin{center}
% \hrulefill\\
% \vspace{.5cm} {\Large \titulo}\\
% \vspace{1.3cm}
% \textbf{\it \autor}\\
% \vspace{.5cm}
% \hrulefill\\
% \end{center}
% \end{minipage}
% \end{center}
%
% \vfill
%
% \cleardoublepage
%
%
% \begin{flushright}
% \begin{Sbox}
% \begin{minipage}{8.5cm}
% \footnotesize
% SERVIÇO DE  PÓS-GRADUAÇÃO DO ICMC-USP\\
% \\
% Data de Depósito: \\
% \\
% Assinatura:\hrulefill
% \end{minipage}
% \end{Sbox}
% \fbox{\TheSbox}
% \end{flushright}
%
%
% \vspace*{2cm}
% \begin{center}
% {\huge\sf \titulo}\footnote{\comentario}
%
%
% \vspace*{2cm}
%
% {\it \autor}
%
% \vspace*{2cm}
%
%
% {\bf Orientador:}  {\it \orientador}
%
% \end{center}
%
% \vspace*{4cm}
%
% \begin{flushright}
% \begin{minipage}{10cm}
% \nota
% \end{minipage}
% \end{flushright}
%
% \vspace*{2cm}
% \begin{center}
% \textbf{USP - São Carlos \\\data}
% \end{center}
% \cleardoublepage



\end{titlepage}
 \tikzexternalize
\pagestyle{plain}
\onehalfspacing
\chapter*{Resumo}
\esb{agradecimentos vai s� na vers�o revisada?}

\ano{arrumar v�rgulas e pontos nos n�meros}

\tar{poss�vel novo t�tulo:}

\tar{Viabilidade do aprendizado de m�quina ativo em vista da diversidade de
estrat�gias}

\ano{citar agencias de fomento? quais? onde?}

\tar{remover cita��es de baixa qualidade no documento todo}

 \tar{escrever sobre implementa��o (apendice?):
 URL, diagrama de classes, LOC, custos totais CPU/mem�ria/disco, total de
registros no BD}

\tar{aprendiz / algoritmo de aprendizado / classificador}

Apesar do crescente avan�o no desenvolvimento de algoritmos capazes de induzir modelos preditivos
numa grande diversidade de dom�nios, entraves de cunho econ�mico, dentre outros, ainda persistem.
Embora tais modelos sejam constru�dos sem programa��o expl�cita,
eles n�o podem, em geral, dispensar a supervis�o humana no processo inicial de descoberta de r�tulos que
normalmente se atribuem a certas parcelas dos dados dispon�veis.
Em vista da forte possibilidade dos dados serem massivos,
as parcelas n�o podem ser proporcionais ao total de dados.
Elas devem permanecer pequenas, dentro do devido or�amento.
Isso prop�e um compromisso entre o custo de rotula��o e a acur�cia preditiva,
cuja solu��o pode ser a ado��o de uma t�cnica de aprendizado ativo.

Existem diversas abordagens de se amostrarem ativamente os dados para rotula��o.
Diferentemente do aprendizado passivo, n�o � poss�vel test�-las para posterior escolha da melhor
segundo alguma m�trica de acur�cia.
A cada r�tulo obtido incorre-se em custo adicional.
Dessa forma, a situa��o ideal seria a ci�ncia pr�via da melhor estrat�gia
para o dom�nio em quest�o, mesmo com poucos r�tulos conhecidos de antem�o;
ou, em maior detalhe,
para cada instante durante a aquisi��o de r�tulos naquele dom�nio.

Nesta tese, a viabilidade do aprendizado ativo � comprovada empiricamente com
solidez estat�stica 
de acordo com tr�s pontos de vista: adapta��o e compara��o dos principais
paradigmas e de sua efetividade em geral;
na defini��o de nichos adequados para cada estrat�gia;
e, na demonstra��o de que � poss�vel definir previamente as estrat�gias mais
adequadas automaticamente. \ano{conferir}
� tamb�m empreendida uma an�lise amplamente negligenciada pela
literatura da �rea:
o risco devido � variabilidade dos algoritmos.

% o vi�s de amostragem tem seus perigos, ent�o quando n�o h� diferen�a estat�stica entre aleat�rio e AA, deve-se prefirir aleat�rio,
% ou melhor, cluster-based, por suas garantias estatisticas peculiares.

\newpage
\thispagestyle{empty}
\mbox{}
\cleardoublepage

\tableofcontents
\cleardoublepage

\listoffigures
\addcontentsline{toc}{section}{Lista de Figuras}
\cleardoublepage

\listoftables
\addcontentsline{toc}{section}{Lista de Tabelas}
\cleardoublepage

\chapter*{Abreviaturas}
\addcontentsline{toc}{section}{Abreviaturas} %ver comando do rigolin para criar links e outras
\begin{acronym}
\acro{$ID$}{densidade de informação}
% $TU$ utilidade de treinamento
% $d(\bm{x},\bm{u})$ medida de distância
% $sim(\bm{x},\bm{u})$ medida de similaridade
\end{acronym}
\cleardoublepage

\listofalgorithms
\addcontentsline{toc}{section}{Lista de Algoritmos}
\cleardoublepage

\hypersetup{pageanchor=true}
\pagenumbering{arabic}
\pagestyle{fancy}
\renewcommand{\chaptermark}[1]{\markboth{\thechapter \ #1}{}}
\renewcommand{\sectionmark}[1]{\markright{\thesection \ #1}}

\usetikzlibrary{trees}

%Capitulos
\input introducao
\input contexto
\input propostas
\input metodologia
\input experimentos
\input exp-analise
\input exp-meta
\input conclusao

\appendix
\input apendice1
\input apendice2
\input apendice3
\input apendice4
\input apendice5
\bibliographystyle{apalike-br}
% \bibliographystyle{abnt-alf}
\bibliography{bibliografia}
\addcontentsline{toc}{chapter}{Referências Bibliográficas}
\end{document}