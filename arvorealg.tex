\afterpage{\clearpage\begin{landscape}
\begin{figure}
\tikzset{
every node/.style={font=\scriptsize,black, thin},
decision/.style={shape=rectangle, minimum height=1cm, text width=1.7cm,
text centered, rounded corners=1ex, draw},
outcome/.style={ shape=rectangle, fill=gray!15, draw, text width=2cm, text justified},
decision tree/.style={sibling distance=3cm, level distance=2cm},
cond/.style={blue, yshift=-2mm, shape=rectangle, text centered},
}
\begin{center}
\caption{Distribuição das estratégias vencedoras de acordo com o algoritmo usado e
características de cada base.
\textit{As folhas apresentam a frequência de vitórias das estratégias.
Uma colocação entre as três melhores é contabilizada como vitória.
As $25\%$ menos frequentes são representadas pela categoria ``demais''.
Mínimo de $40$ exemplos por folha.}
}
\begin{tikzpicture} [edge from parent/.style={->,above,draw,sloped,midway,gray!30,ultra thick},
text width=2.7cm, align=flush center, grow cyclic,
level 1/.style={level distance=3.3cm,sibling angle=60},
level 2/.style={text width=2cm, font=\footnotesize, level distance=3.3cm,sibling angle=60},
level 3/.style={text width=2cm, font=\footnotesize, level distance=3.2cm,sibling angle=60},
level 4/.style={text width=2cm, font=\footnotesize, level distance=3.1cm,sibling angle=45},
level 5/.style={text width=2cm, font=\footnotesize, level distance=2.85cm,sibling angle=45},
]
\input treeLearnerfalse
\label{tree1}
\end{tikzpicture}
\end{center}
\end{figure}
\end{landscape}\clearpage}



