\chapter{Contexto} \label{contexto}
Neste capítulo, são introduzidos os temas pertinentes a esta tese.
Na Seção \ref{apsup},
a classificação em aprendizado de máquina é apresentada.
O assunto central, que engloba o aprendizado ativo e suas principais estratégias,
é introduzido nas seções \ref{aprendizado-ativo} e \ref{estrategias} respectivamente.
A Seção \ref{notacao} antecede a exposição das estratégias visando
definir a notação adotada na descrição delas e no restante do documento.
Na Seção \ref{elmorig}, as \elms são especificamente detalhadas por serem os algoritmos
de aprendizado mais recentes dentre os empregados, pela sua ausência na literatura de
aprendizado ativo e pela adaptação de uma estratégia específica para ela (Capítulo \ref{propostas}).
Finalmente, na Seção \ref{meta}, é feita uma breve revisão sobre meta-aprendizado.
\ano{os outros classificadores vão ser mencionados
neste capítulo ou só a partir do cap. de metodologia?}
% O foco deste trabalho está no problema de se lidar com a variedade
% existente de estratégias de aprendizado ativo e suas eventuais
% afinidades com diferentes domínios de aplicação e algoritmos de aprendizado.
% 
% Com esse intuito,
% vários algoritmos comumente usados para classificação foram adotados.
% Adicionalmente,
% um tipo de algoritmo, até onde alcança o conhecimento do autor,
% nunca utilizado para aprendizado ativo foi adotado.
% Esse algoritmo, chamado de \elm, é descrito na Seção \ref{elm}
% juntamente com algumas de suas variações.
% Complementarmente à tarefa de escolha manual da estratégia mais adequada para um
% dado problema, um esquema automático de recomendação de estratégias
% baseado em meta-aprendizado foi empregado - abordagem possivelmente sem
% precedentes na literatura.
% Dessa forma, uma introdução ao meta-aprendizado é o assunto da
% Seção \ref{meta}.

\section{Classificação}\label{apsup}
Algumas habilidades humanas vêm claramente programadas geneticamente
e outras são apenas a execução de passos memorizados,
como é, respectivamente, o caso da amamentação e o trabalho efetuado numa
determinada posição de uma linha de montagem industrial.
Apesar do amplo escopo de ação desses dois tipos de conhecimento,
ainda resta um vasto espectro de novas habilidades adquiríveis
sem que seja necessária uma programação prévia.
Dois exemplos desse espectro são a alfabetização e
o reconhecimento de novos objetos.
Trata-se de habilidades aprendidas por humanos em que é difícil
identificar os algoritmos nos quais se baseiam.
A identificação da categoria de determinado objeto de interesse
é parte importante das potencialidades
de agentes inteligentes e é chamada de \textit{classificação}.
Essa, por sua vez, depende de um \textit{modelo} baseado nos
dados fornecidos por um supervisor.
A indução desse modelo ocorre por meio de um processo de \textit{aprendizado}.

\textit{Aprendizado de máquina} é a área que engloba algoritmos capazes de
aprender - construir modelos - sem
programação explícita \citep{journals/cacm/Valiant84}.
Diversas tarefas como \ano{tarefa de máquina 1, tarefa2, ... REFERENCIA}
têm sido bem sucedidas atualmente.
Entretanto, durante a construção do modelo, é desejável saber
que parte dos dados é relevante.
Esse é um problema em aberto que pode ser entendido como uma busca
pela melhor maneira de amostragem dos dados
na etapa que antecede o treinamento do algoritmo.
A motivação para essa busca pode frequentemente advir
da intratabilidade de um excesso de dados
ou da escassez de recursos para manutenção de um especialista no
papel de supervisor.
Ambos os casos podem se beneficiar do tipo de
aprendizado denominado \textit{ativo} - apresentado
na Seção \ref{aprendizado-ativo}.

 \input active-learning
 \input elm
 \input meta
\section{Considerações}
 