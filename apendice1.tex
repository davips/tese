\chapter{Cenários}\label{cenarios}
\ano{revisar completamente}

Existem três principais cenários na literatura de aprendizado ativo \citep{settles2010active}:
\textit{membership query synthesis}\footnote{[síntese de consulta por associação ou consulta de
exemplos sintetizados]};
amostragem baseada em \textit{\pool}; e
\textit{amostragem seletiva baseada em fluxo}\footnote{[\textit{stream-based selective sampling}]}.
Eles se dividem nos cenários apresentados com as respectivas estratégias na Tabela
\ref{tab:compara-cenarios-ativos}.
% \setlength{\tabcolsep}{0pt}
% \left|{\mathcal{U}}\right|$
\begin{table}[h]
\begin{center}
\begin{tabular}{|l|c|c|}
%   \cline{2-13}
%   \multicolumn{1}{l|}{} &
%   \multicolumn{4}{c|}{2011} &
%   \multicolumn{4}{c|}{2012} &
%   \multicolumn{4}{c|}{2013} \\
  \hline
  \textbf{cenário} & \textbf{característica} & \textbf{estratégias} \\
                   & \textbf{do repositório} & \textbf{aplicáveis} \ano{colocar direto os nomes das estratégias} \\
  \hline
  \textit{membership query} & $|\mathcal{U}|=0$ & \textit{não há amostragem} \\
  \textit{synthesis}        &                   &  \\
  \hline
  amostragem seletiva baseada & $|\mathcal{U}|=1$ & informatividade isolada \\
  em fluxo - estrita          &                   &   \\
  \hline
  amostragem seletiva baseada & $1 < |\mathcal{U}| < |\mathcal{D}|$ & informatividade isolada \\
  em fluxo - em blocos        &                                     & informatividade conjunta \\
                              &                               & informativo-representatividade \\
  \hline
  \textit{pool-based sampling} & $\mathcal{U} = \mathcal{D}$ & informatividade isolada \\
                               &                               & informatividade conjunta \\
                               &                               & informativo-representatividade \\
  \hline
\end{tabular}
\end{center}
\caption{Comparação de cenários de aprendizado ativo.}
\label{tab:compara-cenarios-ativos}
\end{table}

%%%%%%%%%%%%%%%%%%%%%%%%%%%%%%%%%%%%%%%%%%%%%%%%%%%%%%%%%%%%%%%%%%%%%%%%%%%%%%%%%%%%%
\section*{\textit{Membership query synthesis}}
\ano{procurar por ? em todo o documento}

No cenário de \textit{membership query synthesis}, exemplos são criados pelo
algoritmo de aprendizado \citep{angluin1988queries}.
A criação de um exemplo é ilustrada na Figura \ref{fig:memquerysyn}.
Ele é feito de maneira a otimizar a busca da hipótese ideal dentro do
\textit{version space}\footnote{[espaço de versões]} conforme definido por
\cite{mitchell1997machine} embora nem
sempre sejam criados exemplos que efetivamente ocorreriam na aplicação - alguns não seriam
pertencentes à distribuição natural dos dados.
Trata-se de uma abordagem viável para problemas de domínio finito cujos atributos tenham valores
semanticamente interpretáveis pelo oráculo.
A interpretabilidade é importante para que um oráculo humano seja capaz de ponderar a respeito
dos
exemplos que lhe são apresentados.
Assim, aplicações como reconhecimento de escrita ou processamento de linguagem natural podem ser
inadequadas para um aprendizado baseado na síntese de exemplos.
No caso do reconhecimento de escrita, há pelo menos um relato de caracteres híbridos terem sido
gerados pelo aprendiz ativo de forma que não pudessem ser reconhecidos - e eventualmente
rotulados
- por humanos \citep{settles2010active,baum1992query}. Em outros domínios em que o oráculo não é
humano, como um teste químico ou um robô seguindo coordenadas, \textit{membership querysynthesis}
mostra-se uma abordagem promissora \citep{cohn1996active}.

\begin{figure} %[H]
    \centering
    \scalebox{.75}{\begin{tikzpicture}[node distance=0, auto, transform shape,remember picture]
% 	\tikzset{node distance=1, label distance=-5}
\node (e) [, b] {
  \begin{tabular}{ccc}
    \textit{membership}\\
    \textit{query}\\
    \textit{synthesis}
  \end{tabular}
};
\node (x)  [left= 1 of e,draw, circle, inner sep=2, thick, top color=orange!30, bottom color=orange!30, middle color=white] {$\langle \bm{x},\_\rangle$};
\homem [h,] {left=1 of x, label=below:oráculo}
\node (xy) [left=2 of h,draw, circle, inner sep=2, thick, top color=blue!30, bottom color=blue!30, middle color=white] {$\langle \bm{x},y\rangle$};
\modelovs [m,, { }] {below = 2 of h.south}

\draw[-, setatr] (h.west) to [out=170,in=10] node[] {} (xy);
\draw[->, setatr] (xy.south) to [out=-100, in=100] node[] {} (am.north);
\draw[->, setats] (x.west) to  node[] {} (h.east);
\draw[-, setats] (e) to  node[] {} (x);
\draw[->, seta] (cm.north) to [out=90, in=-90] node[] {$\bm{x}$} (e.south);
\end{tikzpicture}
}
    \caption{\textit{Membership query synthesis}: as consultas ao oráculo são criadas sob medida.}
    \label{fig:memquerysyn}
\end{figure}

\section*{\textit{Pool-based sampling}}
Muitos problemas proveem grande quantidade de exemplos de uma só vez. Eles são a motivação da
\textit{pool-based sampling} \citep{lewis1994heterogeneous}.
Esse tipo de amostragem assume que há apenas uma pequena parcela de exemplos rotulados e que
normalmente é possível analisar todos os exemplos não rotulados a qualquer momento, pois estariam
disponíveis num repositório conforme pode ser visto na Figura \ref{fig:poolbased}.
Uma análise típica consiste em extrair alguma medida informativa de cada exemplo.

\begin{figure} %[H]
    \centering
    \begin{tikzpicture}[node distance=0, auto, transform shape,remember picture]
% 	\tikzset{node distance=1, label distance=-5}
\cts [ts,] {thick}
\node (x)  [below right= 1 of ts,draw, circle, inner sep=2, thick, top color=orange!30, bottom color=orange!30, middle color=white] {$\langle \bm{x},\_\rangle$};
\node (e) [below left=1 of ts, b] {
  \begin{tabular}{ccc}
    \textit{pool-based}\\
    \textit{sampling}
  \end{tabular}
};
\homem [h,] {left=1 of e, label=below:or�culo}
\node (xy) [left=2 of h,draw, circle, inner sep=2, thick, top color=blue!30, bottom color=blue!30, middle color=white] {$\langle \bm{x},y\rangle$};
\modeloprob [m,, { }] {below = 2 of h.south}

\draw[-, setatr] (h.west) to [out=170,in=10] node[] {} (xy);
\draw[->, setatr] (xy.south) to [out=-100, in=100] node[] {} (am.north);
\draw[->, setats] (x.west) to  node[] {} (e.east);
\draw[->, setats] (e) to  node[] {$\bm{x}$} (h);
\draw[-, setats] (ts.east) to [out=-10, in=100] node[] {} (x.north);
\draw[->, setats] (x.south) to [out=-90, in=30] node[] {} (cm.east);
\draw[->, seta] (cm.north) to [out=90, in=-90] node[] {$m(\bm{x})$} (e.south);
\draw[<->, setats] (ts.west) to [out=180, in=90] node[] {} (e.north);
\end{tikzpicture}

    \caption{\textit{Pool-based sampling}: o aprendiz ativo tem livre acesso ao repositório.}
    \label{fig:poolbased}
\end{figure}

Muitas aplicações são adequadas a esse cenário, sendo que a classificação de textos
\citep{mccallum1998employing} tem recebido maior destaque por requerer esforço humano na
rotulação
dos textos.
Um exemplo de aplicação em que o oráculo é uma máquina, é caso do meta-aprendizado para escolha
do
melhor algoritmo para cada tipo de base de dados.
Nesse caso, cada base corresponde a um meta-exemplo e o rótulo indica qual o algoritmo de aprendizado mais
adequado para ela.
Por isso, o rótulo é obtido apenas depois de todos os classificadores terem sido treinados e
testados nela.
Além disso, essas bases podem ser muito volumosas, reforçando a necessidade de aplicação do
aprendizado ativo
na seleção dos metaexemplos tal como feito por \cite{prudencio2007active}.



%%%%%%%%%%%%%%%%%%%%%%%%%%%%%%%%%%%%%%%%%%%%%%%%%%%%%%%%%%%%%%%%%%%%%%%%%%%%%%%%%%%%%%%5
\section*{Amostragem seletiva baseada em fluxo} \label{sec:cenario_fluxo}
Os fluxos de dados, conforme definição da Seção \ref{sec:fluxo:def}, precisam ser tratados dentro
de limitações de tempo de processamento e de espaço em memória.
Logo, a manipulação de um repositório, ainda que ajustado à memória disponível, pode ser inviável
devido à necessidade de uma rápida resposta a cada um dos exemplos que chegam incessantemente.
Nesse tipo de configuração se situa a amostragem seletiva baseada em fluxo
\citep{cohn1994improving}. %essa citacao do settles não faz lá muito sentido
Nela, exemplos são obtidos um a um de um fluxo e imediatamente descartados ou enviados para o
oráculo.
Optou-se neste documento por complementar o nome desse cenário com o termo \textit{estrito}.
A consequência dessa política de consulta ``agora ou nunca'' é um uso minimalista de recursos
computacionais durante o processo decisório a respeito da necessidade de rotulação de cada
exemplo.
A amostragem baseada em fluxo é ilustrada na Figura \ref{fig:fluxoquery}.

\begin{figure} %[H]
    \centering
    \begin{tikzpicture}[node distance=0, auto, transform shape,remember picture]
% 	\tikzset{node distance=1, label distance=-5}
\fluxo [ts,] {thick}
\node (x)  [below= 1 of ts,draw, circle, inner sep=2, thick, top color=orange!30, bottom color=orange!30, middle color=white] {$\langle \bm{x},\_\rangle$};
\node (e) [left=1 of x, b] {
  \begin{tabular}{ccc}
    \textit{amostragem}\\
    \textit{seletiva}\\
    \textit{baseada em fluxo}
  \end{tabular}
};
\homem [h,] {left=1 of e, label=below:oráculo}
\node (xy) [left=2 of h,draw, circle, inner sep=2, thick, top color=blue!30, bottom color=blue!30, middle color=white] {$\langle \bm{x},y\rangle$};
\modeloprob [m,, { }] {below = 2 of h.south}

\draw[-, setatr] (h.west) to [out=170,in=10] node[] {} (xy);
\draw[->, setatr] (xy.south) to [out=-100, in=100] node[] {} (am.north);
\draw[->, setats] (x.west) to  node[] {} (e.east);
\draw[->, setats] (e) to  node[] {$\bm{x}$} (h);
\draw[-, setats] (ts.south) to [out=-10, in=100] node[] {} (x.north);
\draw[->, setats] (x.south) to [out=-90, in=30] node[] {} (cm.east);
\draw[->, seta] (cm.north) to [out=90, in=-90] node[] {$m(\bm{x})$} (e.south);

\node (descarte) [above=1 of e] {descarte};
\draw[->, setats] (e.north) to [out=90, in=-90] node[] {$\bm{x}$} (descarte.south);
\end{tikzpicture}

    \caption{Amostragem seletiva baseada em fluxo: todo novo exemplo não consultado precisa ser
descartado.}
    \label{fig:fluxoquery}
\end{figure}

É importante observar a diferenciação feita por \cite{zliobaite2011active} entre
\textit{aprendizado ativo online}\footnote{[\textit{online active learning}]
}
 e \textit{aprendizado ativo em fluxo de dados}\footnote{[\textit{active learning in
datastreams}]
}
- embora ambos sejam aplicáveis em ambientes com
\underline{relevantes restrições}\footnote{As restrições são relevantes conforme explicado na
Seção
\ref{sec:fluxo:def}.
} de recursos:
em fluxos de dados é esperada a ocorrência de mudanças de conceito.

Uma variação da amostragem baseada em fluxo
é o cenário em que ele é composto por \textit{blocos de dados}\footnote{[\textit{data chunks}]
} (ou seja, \textit{não estrito}).
Isso permite o tratamento de cada bloco com uma \textit{pool-based sampling}.
É o caso do trabalho de \cite{zhu2007active}.
Esse cenário exige adaptações nas formas convencionais de consulta e juntamente com elas é
apresentado na Seção \ref{sec:panorama}.