% %%%%%%%%%%%%%%%%%%%%%%%%%%%%%%%%%%%%%%%%%%%%%%%%%%%%%%%%%%%%%%%%%%%%%%%%%%%%%%%
\chapter{Formata��o e estilo}\label{estilo}
Algumas regras foram adotadas para maior consist�ncia do texto.

Siglas e s�mbolos s�o empregados conforme o uso mais comum na literatura pesquisada.
Por exemplo, uma vari�vel representada por uma letra � mnem�nica a uma palavra inglesa - ou de outro idioma se for o caso do termo original.
Os termos que introduzem um t�pico n�o merecedor de uma se��o pr�pria aparecem em negrito.

As notas de rodap� s�o dispens�veis para o entendimento do texto, por�m elas foram adotadas nos casos a seguir para facilitar o escrut�nio quando desejado.
Quando poss�vel, os termos em ingl�s foram traduzidos para o portugu�s sem perda do sentido original.
Na primeira ocorr�ncia de cada tradu��o, ela aparece escrita em it�lico (quando j� n�o estiver destacada de outra forma por outro motivo) com os seus respectivos termos originais dados em nota de rodap�.
Termos com tradu��o incerta aparecem sempre em it�lico, mas com uma sugest�o de tradu��o em nota de rodap� apenas na primeira ocorr�ncia.
Nesse caso, as notas de rodap� s�o dadas em min�sculas e entre colchetes.
Express�es sublinhadas indicam a exist�ncia de maiores detalhes em notas de rodap�.
%
