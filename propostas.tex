\chapter{Contribui��o} \label{cap:contrib}

\section{Adapta��es de Classificadores}
\subsection{IELM}
\subsection{EIELM}
\subsection{CIELM}
\subsection{nintera}
\blue{citar apendice}

% % % % % % % % % % % % % % % % % % % % % % % % % % % % % % % % % % % % % % % % % % % % 
\section{Propostas}

\subsection{Busca no espa�o de hip�teses multiclasse}
SGmulti

% \subsection{SVMmulti} %parece que foi um fiasco quando desbalanceado (pior que random)

\subsection{Extens�es da amostragem ponderada por densidade}

\subsubsection{DWLaU}% - Amostragem ponderada por densidade e utilidade do r�tulo}

\subsubsection{DWLoU}% - Amostragem ponderada por densidade e utilidade local}

\subsubsection{DWLoLaU}% - Amostragem ponderada por densidade e utilidade local do r�tulo}
 \blue{consertar e rodar DWLoLaU, caso DWLoU e DWLaU sejam bons; excluir t�pico caso
negativo}
 
\subsection{Meta-estrat�gia din�mica}

% % % % % % % % % % % % % % % % % % % % % % % % % % % % % % % % % % % % % % % % % % % %
\section{Viabilidade do aprendizado ativo}
% citar: contra random; compara��o 1-a-1; variabilidade (risco); tolerable waiting time

\subsection{Efetividade geral}
\blue{friedman um a um n�o est� dando nada?}
\blue{friedman talvez funcione melhor por nicho (metaclusters)}

\subsection{Afinidades estrat�gia-dom�nio-classificador-or�amento}

\subsection{Risco}
\subsection{Tempo de espera toler�vel}
